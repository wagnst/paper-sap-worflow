%%%%% 
% 
% WORKFLOW MODUL TINF13AIBC
% S. Wagner, M. Doerfler, J. Dann / SAP AG, Juni 2014
%
%%%%%

\documentclass[12pt,pdftex,a4paper,oneside]{scrreprt} 

% Settings of TeX Document
%%%
%
% haeufig benoetigte Festlegungen
%
%%%

%% Erweiterungspakete werden geladen

% erlaubt direkte Verwendung von Umlauten im Quelltext (latin1)
%\usepackage{umlaut} 
\usepackage[utf8]{inputenc}% http://ctan.org/pkg/inputenc

% länderspezifische Einstellungen
% sorgt u.a. für die korrekte Silbentrennung,
% deutsche Bezeichner ("Tabelle", "Abbildung" etc.)
\usepackage[ngerman]{babel}

% fuer deutsche W"orter in der Literaturliste
\usepackage{bibgerm}  

% Sonderzeichen, z.B. Euro mit \texteuro   
\usepackage{textcomp}         

% erlaubt einfache Auswahl der Art der Nummerierung bei Aufzählungen
\usepackage{enumerate}

% erlaubt das Einbinden von Grafiken
\usepackage{graphicx}

% erlaubt die einfache Aenderung der Seitenraennder
%\usepackage[left=15mm,right=15mm,top=19mm,bottom=19mm]{geometry}

\usepackage[ngerman]{translator}

\usepackage{tabularx}    
                       
\usepackage{booktabs}  

\usepackage{setspace} 

\usepackage{amssymb}

\usepackage{eurosym}

\usepackage{lastpage}

\usepackage{multicol}

\usepackage{pgf-pie,etoolbox}

\usepackage{wrapfig}

\usepackage{float}

\usepackage{listings}

\definecolor{light-gray}{gray}{0.95}
% ------------------------------ Geometrie--------------------------------------
\usepackage{geometry,blindtext}
\geometry{a4paper,left=30mm,right=20mm, top=1cm, bottom=2cm, includeheadfoot}

% ---------------------------- Kopf-Fußzeilen-----------------------------------
\usepackage{fancyhdr}
\pagestyle{fancy}
\fancyhf{}

\fancyhead[L]{\small{\textbf{Proseminar Workflow}}}
\fancyhead[C]{\small{Steffen Wagner, Marco Dörfer, Jonas Dann}}
\fancyhead[R]{\includegraphics[scale=0.1]{grafiken/sap_logo.png}}
\renewcommand{\headrulewidth}{0.5pt} %obere Trennlinie
\fancyfoot[C]{\thepage\ von \pageref{LastPage}} %Seitennummer
\renewcommand{\footrulewidth}{0.4pt} %untere Trennlinie
\def\chapterpagestyle{fancy} %auch bei Seiten mit Chapter Kopfzeile anzeigen


% ----------------------------- Hyperlinks -------------------------------------
\usepackage{breakurl}         % Zeilenumbruch f"ur URLs

% Links zum Anklicken im DVI- und PDF-Dokument
\usepackage{hyperref} 
\hypersetup{colorlinks
  ,linkcolor=blue             % toc, Glossar-Begriffe, Seitenzahlen in Index und Glossar
  ,urlcolor=blue              % URLs, die mit \url{} erzeugt wurden
  ,citecolor=blue             % Literatur-Zitate, die mit \cite erzeugt wurden
  ,filecolor=red              % Verweise auf Dateien, hier nicht verwendet
  ,breaklinks=true            % Zeilenumbruch f"ur Links
  ,linktocpage                % Nur Seitenzahlen sind Links, nicht ganze Zeilen
}
\def\UrlFont{\sffamily} 

% ------------------------------ Glossar ---------------------------------------
\usepackage[
% nonumberlist,               % keine Seitenzahlen anzeigen
acronym,                      % ein Abk"urzungsverzeichnis erstellen
toc,                          % Eintr"age im Inhaltsverzeichnis
section                       % im Inhaltsverzeichnis auf Section-Ebene erscheinen
]{glossaries}                 % definiert den Befehl \printglossary
\renewcommand*{\glspostdescription}{} %Den Punkt am Ende jeder Beschreibung deaktivieren
 
%Ein eigenes Symbolverzeichnis erstellen
\newglossary[slg]{symbolslist}{syi}{syg}{Symbolverzeichnis}

%Glossar-Befehle anschalten
\makeglossaries
 
% ----------------------------- Stichwortverzeichnis ---------------------------
\usepackage{makeidx}          % definiert den Befehl \printindex
\makeindex                    % erzeugt fuenftes.idx f"ur den Index

% ----------------------------- Aussehen einer Seite ---------------------------
%\textheight240mm              % Hoehe des Textes
%\textwidth150mm               % Breite des Textes
%\topmargin-20mm               % oberer Rand
%\oddsidemargin-7mm            % linker Rand bei ungeraden Seitenzahlen
%\evensidemargin-7mm           % linker Rand bei geraden Seitenzahlen
%\pagestyle{plain}             % plain    = Seitenzahlen, aber keine Kopfzeilen
                              % empty    = ohne Seitenzahlen
                              % headings = mit Kopfzeilen
%\parindent0mm                 % kein Einr"ucken am Anfang eines Absatzes


% kein "haengender" Einzug der ersten Zeilen eines Absatzes
\setlength{\parindent}{0cm}

% vertikaler Abstand zwischen Absätzen
% (1ex entspricht der Höhe des Buchstabens x, diese Angabe ist
% relativ zur gewählten Schriftgröße und passt sich somit bei
% einer Änderung entsprechend an)
\setlength{\parskip}{1ex}



% Alle vorkommenden Abkürzungen oder wichtige Begriffe
%GLOSSAR

%glossaries mit Acronymen
\newglossaryentry{abap}
{
    name={ABAP},
    description={Ist eine Programmiersprache der \gls{sap} AG.},
    first={Advanced Business Application Programming (ABAP)},
    long={Advanced Business Application Programming}
}

\newglossaryentry{ui}
{
    name={UI},
    description={Bezeichnet die Bedienoberfläche eines Computerprogramms.},
    first={User Interface (UI)},
    long={User Interface}
}

\newglossaryentry{nw}
{
    name={NW},
    description={SAP NetWeaver ist ein Produkt der Firma SAP, die NetWeaver als Plattform für Geschäftsanwendungen bezeichnet. Grundlage für alle Anwendungen auf NetWeaver ist der SAP NetWeaver Application Server.},
    first={NetWeaver (NW)},
    long={NetWeaver}
}

\newglossaryentry{erp}{
		name=ERP,
		description={Ein Enterprise-Resource-Planning-System (ERP-System) unterstützt sämtliche in einem Unternehmen ablaufenden Geschäftsprozesse. Es enthält Module für die Bereiche Beschaffung, Produktion, Vertrieb, Anlagenwirtschaft, Personalwesen, Finanz- und Rechnungswesen usw., die über eine gemeinsame Datenbasis miteinander verbunden sind.},
		first={Enterprise Resource Planning (ERP)},
		long={Enterprise Resource Planning}
}

\newglossaryentry{crm}{
		name=CRM,
		description={CRM ist zu verstehen als ein strategischer Ansatz, der zur vollständigen Planung, Steuerung und Durchführung aller interaktiven Prozesse mit den Kunden genutzt wird. CRM umfasst das gesamte Unternehmen und den gesamten Kundenlebenszyklus und beinhaltet das Database Marketing und entsprechende CRM-Software als Steuerungsinstrument},
		first={Customer Relationship Managment (CRM)},
		long={Customer Relationship Managment}		
}

\newglossaryentry{srm}{
		name=SRM,
		description={Einkaufspolitik ist ein Teilgebiet der Unternehmenspolitik, das sich mit der Bestimmung von Zielen des Einkaufs und der Festlegung von Instrumenten zur Zielverwirklichung befasst. Wesentliche Ziele der Einkaufspolitik sind die Sicherung der Versorgung mit dem in quantitativer und qualitativer Hinsicht richtigen Material sowie die Minimierung der damit verbundenen Kosten},
		first={Supplier Relationship Managemen (SRM)},
		long={Supplier Relationship Managemeng}		
}

\newglossaryentry{scm}{
		name=SCM,
		description={Supply Chain Management bezeichnet den Aufbau und die Verwaltung integrierter Logistikketten (Material- und Informationsflüsse) über den gesamten Wertschöpfungsprozess, ausgehend von der Rohstoffgewinnung über die Veredelungsstufen bis hin zum Endverbraucher},
		first={Supply Chain Management (SCM)},
		long={Supply Chain Management}		
}

\newglossaryentry{plm}{
		name=PLM,
		description={Produktlebenszyklus-Management oder kurz PLM ist ein Hebel für eine erfolgreiche Produktentwicklung und ein strategischer Faktor, der im gesamten Unternehmen zum wirtschaftlichen Nutzen beiträgt. Mithilfe von PLM können Unternehmen komplexe, funktionsübergreifende Prozesse steuern und die Arbeit verteilter Teams so koordinieren, dass konsistent und effizient die bestmöglichen Produkte entstehen},
		first={Produkt Lifecycle Management (PLM)},
		long={Produkt Lifecycle Management}		
}

\newglossaryentry{byd}{
		name=ByD,
		description={Business By Design},
		first={Business By Design (ByD)},
		long={Business By Design}		
}

\newglossaryentry{saas}{
		name=SaaS,
		description={Software-as-a-Service},
		first={Software-as-a-Service  (SaaS)},
		long={Software-as-a-Service}		
}

\newglossaryentry{sme}{
		name=SME,
		description={Small and medium enterprises},
		first={Small and medium enterprises  (SME)},
		long={Small and medium enterprises}		
}

%glossaries nur Worterklärung
\newglossaryentry{sap}{
name=SAP,
description={Systems Applications Products / Systeme Anwendungen Produkte}
}

\newglossaryentry{ibm}{
name=IBM,
description={International Business Machines Corporation}
}

\newglossaryentry{hana}{
name=HANA,
description={High Performance Analytic Appliance}
}

\newglossaryentry{db}{
name=DB,
description={Datenbank}
}

\newglossaryentry{bzw}{
name=Bzw.,
description={Beziehungsweise}
}

\newglossaryentry{ua}{
name=u.a.,
description={unter anderem}
}

\newglossaryentry{vgl}{
name=Vgl.,
description={Vergleich}
}

\newglossaryentry{zb}{
name=z.B.,
description={zum Beispiel}
}









%%%%%
%% jetzt beginnt das Dokument
\begin{document}
%%%%%

% Titelseite
\thispagestyle{empty}

\includegraphics[scale=0.5]{grafiken/sap_logo.png} \hfill \includegraphics[scale=0.2]{grafiken/dhbw_logo.png} \\

\begin{center}

\vspace{0.5cm}
{\large Fakultät Technik - Angewandte Informatik IBC}\\
{\large der Dualen Hochschule Baden-Württemberg Mannheim}\\

\vspace{1.5cm}

{\Large Seminararbeit \\
				Modul \verb|T2INF4122| (Proseminar Workflow) 
}\\

\vspace{1.5cm}
{ \LARGE \bf
        Workflowmanagement anhand von \\ SAP Enterprise Resource Planning \\ und \\ SAP BusinessByDesign\\
}

\vspace{1.5cm}

\begin{tabular}{lll}
Autoren	&:& Steffen Wagner (8974337) \\
&& Marco Dörfler (6541564) \\
&& Jonas Dann (3346893) \\
Kurs   &:& TINF13AIBC \\
Seminarleiter   &:& Kai-Frank Strugalla \\
Bearbeitungszeitraum   &:& 17.05.2014 - 22.06.2014 \\
\end{tabular}


%{\large
\vspace{1.5cm}
Copyright 2014 \\
SAP Aktiengesellschaft \\
Dietmar-Hopp-Allee 16 \\
D-69190 Walldorf \\
\vspace{0.5cm}

\end{center}
\newpage
\thispagestyle{empty}
\mbox{}

% Selbstständigkeitserklärung
\newpage
\thispagestyle{empty}
\section*{Selbstständigkeitserklärung}
\vspace{15mm}
 
Der Verfasser erklärt, dass er die vorliegende Arbeit selbständig, 
ohne fremde Hilfe und ohne Benutzung anderer als der angegebenen Hilfsmittel
angefertigt hat. 
Die aus fremden Quellen (einschließlich elektronischer Quellen) direkt oder 
indirekt übernommenen Gedanken sind ausnahmslos als solche kenntlich gemacht. 
 
%% Abstand und Linie
% 3 Spalten (für alle 3 Leute)
\vspace{3cm} 
Walldorf, den \today
\vspace{1cm} 	
\begin{multicols}{3}
	\vspace{2cm}
	\rule{5cm}{.1pt}\\
	\vspace{5mm}
	Steffen Wagner
	
	\rule{5cm}{.1pt}\\
	\vspace{5mm}
	Jonas Dann

	\rule{5cm}{.1pt}\\
	\vspace{5mm}
	Marco Dörfler
\end{multicols}

\vspace{5cm}

% Abstract 
%Kurze Zusammenfassung
\addchap{Zusammenfassung}
\label{chap:Zusammenfassun}

\begin{tabular}{lll}
Autoren	&:& Steffen Wagner (8974337) \\
&& Marco Dörfler (6541564) \\
&& Jonas Dann (3346893) \\
Telefon	&:& +49 6227 7-56737 \\
Email	&:& steffen.tobias.wagner@sap.com \\
&& jonas.dann@sap.com\\
&& marco.doerfler@sap.com\\
\end{tabular}

\vspace*{3em}

-----------------------\\
\large{FRAGEN+Antworten AN HR. STRUGALLA}\\
- Installation, Konfiguration HANA und ERP als 1/3 der Arbeit (viel Aufwand!)...\verb|->| eventuell Rahmensprengend (nur erwähnen, dass "`es geht"'; Technik von Interesse (NetWeaver Aufbau, Datenbank,...)\\
- ByD \verb|->| kleine Prozesse durchspielen; Grenzen aufzeigen (oder ERP "`intensiver"');|| Workflow Builder!\\
- Prozesse im ERP richtig durchspielen + erstellen (Storyboard, Grafiken machen,..)\\ 
- Wie beeinflusst der Workflow-Builder andere SAP Systeme (CRM, SRM,..) Vorteile, Grenzen!\\
- wie können Legacy Systeme angesprochen werden (evtl. auch mit kleinem Bsp.\\
- Exkurs technisch HANA (kleine Demo Datenselektion)\\



%gruppe ohne Fancy Header / Footer
\begingroup
	\pagestyle{empty} 	
	%Verzeichnisse
	\tableofcontents
	\listoffigures
	\listoftables
	\lstlistoflistings
\endgroup 

%Eigentlicher Inhalt
\chapter{SAP AG} \label{chap:sap}
%%%%%%%%%%%%%%%%%%%%
%% KAPITEL SAP AG %%
%%%%%%%%%%%%%%%%%%%%
Die, 1972 von fünf ehemaligen \gls{ibm}-Mitarbeitern gegründete, \gls{sap} AG ist als weltweit viertgrößter Softwarehersteller (Stand Q4/2013, \cite{SAPFacts}) der Marktführer im Bereich betriebswirtschaftlicher Standardsoftware. Mit weltweit mehr als 66.500
Mitarbeitern (Stand Q4/2013, \cite{SAPAtGlance}) und über 253.500 Kunden in 188 Ländern (Stand Q4/2013, \cite{SAPAtGlance}) erwirtschaftet sie einen jährlichen Umsatz von ca. 16,82 Milliarden \euro (Euro) (Stand Q4/2013, \cite{SAPFacts}). Tabelle \ref{tab:SAPKennzahlen} zeigt die Entwicklung wichtiger Kennzahlen der SAP AG \cite{SpringerControllingSAP}.

%% Tabelle mit SAP Kennzahlen, aktualisiert Juni 2014
% der optionale Parameter "h" gibt an, dass der Block
% mit der Abbildung vorzugsweise an der aktuellen Position,
% alternativ unten ("botton") platziert werden soll
\begin{table}[hb]
\begin{center}
\begin{tabular}{l||l|l|l|l|l|l}
  & \emph{2002} & \emph{2004} & \emph{2006} & \emph{2008} & \emph{2010} & \emph{2013}\\	
  \hline
  Umsatz (in Mio. \euro) & 7.413 & 7.514 & 9.402 & 11.575 & 12.464 & 16.820\\
  \hline
  Betriebsergebnis (in Mio. \euro) & 1.626 & 2.018 & 2.563 & 2.701 & 2.591 & 5.900\\
  \hline
  Mitarbeiter & 28.797 & 32.205 & 39.355 & 51.544 & 53.513 & 66.500\\
  \hline	
\end{tabular}
\end{center}
% Beschriftung festlegen:
\caption{Entwicklung wichtiger Kennzahlen der \gls{sap} AG} 
% ein Label definieren, mit dessen Hilfe man (an beliebiger Stelle im Dokument) Bezug nehmen kann:
\label{tab:SAPKennzahlen}
\end{table}

\gls{sap} erzielt Umsätze nicht nur mit Software. Der Anteil von Software an den Gesamtumsätzen macht lediglich 26\% aus. Daneben spielen insbesondere die Bereiche Support und Beratung eine große Rolle. Abbildung \ref{abb:SAPUmsatzverteilung} zeigt die Verteilung der Umsätze im Jahr 2010 auf einzelne Bereiche der \gls{sap} AG.

%%Kuchendiagramm Verteilung der Umsätze auf einzelne Bereiche der SAP AG 2010
\begin{figure}[h]
  \centering 
  \begin{tikzpicture} 
    \pie[text=legend,radius=2]{49/Support, 26/Software, 18/Beratung, 7/Sonstige} 
  \end{tikzpicture} 
  \caption{Verteilung der Umsätze auf einzelne Bereiche der \gls{sap} AG} 
  \label{abb:SAPUmsatzverteilung} 
\end{figure} 

Neben dem Firmenhauptsitz Walldorf existieren noch Niederlassungen in über 130 Ländern \cite{SAPLocations} rund um den Globus.
Das Produktportfolio der SAP AG enthält Lösungen für alle zentralen Geschäftsabläufe in Firmen. Dazu gehören unter anderem \gls{erp} (siehe \ref{sec:erp-definition}), \gls{crm} (siehe \ref{sec:crm-definition}), \gls{srm} (siehe \ref{sec:srm-definition}), \gls{scm} (siehe \ref{sec:scm-definition}) oder \gls{plm} (siehe \ref{sec:plm-definition}) Systeme.



\chapter{Grundbegriffe}  \label{chap:grundbegriffe}
%%%%%%%%%%%%%%%%%
%% KAPITEL ERP %%
%%%%%%%%%%%%%%%%%
%% STEFFEN
%%%%%%%%%%%%%%%%%
\section{Enterprise Resource Planning}
\label{sec:erp-definition}
Bei \gls{erp} Systemen handelt es sich um betriebswirtschaftliche Software, die in Betrieben oder Unternehmen eingesetzt werden kann. \gls{erp} IT-Systeme stehen für die Systemintegration der gesamten finanz- und warenwirtschaftlich orientierten Wertschöpfungskette. Dabei umfassen sie alle Teilprozesse von der strategischen und operationalen Planung über Herstellung, Distribution bis hin zur Steuerung von Auftragsabwicklung und Bestandsmanagement. Ein derartiges System verknüpft insbesondere Informationen über Finanzen, personelle Ressourcen, Produktion, Vertrieb und Einkauf. Es verbindet Kundendatenbanken, Auftragsverfolgung, Debitoren- und Kreditorenbuchaltung, Lagerverwaltung und vieles mehr \cite{ERPDefinition}.

%% Kuchendiagramm Marktanteile der Softwareunternehmen bei ERP Software
\begin{figure}[H]
  \centering 
  \begin{tikzpicture} 
    \pie[text=legend,radius=2]{46.8/Andere Anbieter, 25.4/SAP, 12.4/Sage, 6/infor, 5/Microsoft, 4.5/Oracle} 
  \end{tikzpicture} 
  \caption{Marktanteile der Softwareunternehmen bei \gls{erp} Software} 
  \label{abb:SAPMarktanteil} 
\end{figure} 

Im Gegensatz zu den Hauptwettbewerbern Oracle und Microsoft konzentriert sich \gls{sap} auf Unternehmenssoftware. Mit ihren \gls{erp}-Produkten erlangt sie weltweit einen Marktanteil von über 25\% (Siehe Abbildung \ref{abb:SAPMarktanteil}).

%%%%%%%%%%%%%%%%%
%% KAPITEL SCM %%
%%%%%%%%%%%%%%%%%
%% STEFFEN
%%%%%%%%%%%%%%%%%
\section{Supply Chain Management}
\label{sec:scm-definition}
Der Ausdruck \gls{scm} bzw. Lieferkettenmanagement, auch Wert-schöpfungslehre, bezeichnet die Planung und das Management aller Aufgaben bei Lieferantenwahl, Beschaffung und Umwandlung sowie aller Aufgaben der Logistik. Insbesondere enthält es die Koordinierung und Zusammenarbeit der beteiligten Partner (Lieferanten, Händler, Logistikdienstleister, Kunden). \gls{scm} integriert Management innerhalb der Grenzen eines Unternehmens und über Unternehmensgrenzen hinweg. Wesentliches Paradigma hierbei ist es, dass nicht mehr Einzelunternehmen, sondern stattdessen vernetzte Lieferketten miteinander konkurrieren, wodurch eine Integration und Koordination der Mitglieder des Systems "`Lieferkette"' nötig wird. Diese Aufgabe übernimmt das \gls{scm} \cite{SCMDefinition}.

%%%%%%%%%%%%%%%%%
%% KAPITEL PLM %%
%%%%%%%%%%%%%%%%%
%% STEFFEN
%%%%%%%%%%%%%%%%%
\section{Product Lifecycle Management}
\label{sec:plm-definition}
\gls{sap} \gls{plm} dient dem Verwalten und Steuern, also dem Orgranisieren und Managen der Aufgaben, die sich aus dem kompletten Produkt "`Lebenszyklus"' ergeben. Es ist also darauf fokusiert Unternehmen beim Organisieren der Entwicklung von neuen Produkten zu Helfen. Von der Konstruktion und Produktion über den Vertrieb bis hin zur Demontage und dem Recycling \cite{PLMDefinition}.

%%%%%%%%%%%%%%%%%
%% KAPITEL SRM %%
%%%%%%%%%%%%%%%%%
%% STEFFEN
%%%%%%%%%%%%%%%%%
\section{Supplier Relationship Management}
\label{sec:srm-definition}
\gls{srm} ist der Bereich des Supply Chain Managements, der sich mit der Auswahl, Steuerung und Kontrolle der Lieferanten beschäftigt und sich auf die spezifischen Anforderungen, die sich aus der Beschaffung von Gütern und Dienstleistungen ergeben, konzentriert. Das Ziel des Lieferantenmanagements ist die effizientere Gestaltung und Koordination der Beziehungen und Prozesse zwischen einer Organisation und deren Lieferanten \cite{SRMDefinition}.

%%%%%%%%%%%%%%%%%
%% KAPITEL CRM %%
%%%%%%%%%%%%%%%%%
%% STEFFEN
%%%%%%%%%%%%%%%%%
\section{Customer Relationship Management}
\label{sec:crm-definition}
\gls{crm} steht für Customer Relationship Management. Es handelt sich um eine bereichs-übergreifende, IT-unterstützte Geschäftsstrategie, die auf den systematischen Aufbau und die Pflege dauerhafter und profitabler Kundenbeziehungen abzielt. Durch dieses System soll der Marktanteil eines Unternehmens erhöht und die Kundenzufriedenheit gesteigert werden. Außerdem soll eine Segmentierung des Kundenstamms erreicht werden. Eine zentrale Erfassung der Daten bietet den Vorteil Kosten zu reduzieren \cite{ERPDefinition}.



\chapter{SAP Produktübersicht}  \label{chap:sap-produkte}
\section{Large Enterprises}
%%%%%%%%%%%%%%%%%%%%%%%%%%%%
%% KAPITEL Business SUITE %%
%%%%%%%%%%%%%%%%%%%%%%%%%%%%
%% STEFFEN
%%%%%%%%%%%%%%%%%%%%%%%%%%%%
\subsection{SAP R/3 Business Suite}
\label{sec:business-suite}
Als \gls{sap} Business Suite wird die Sammlung von Geschäftsanwendungen bezeichnet, welche Informations- und Prozessintegration, Zusammenarbeit, Industrie spezifische Funktionen, sowie Skalierbarkeit anbietet. Die \gls{sap} Business Suite basiert auf \gls{sap} \gls{nw} (Kapitel \ref{sec:netweaver}).
Die Version 7 der Business Suite beinhaltet sieben Komponenten:

\begin{itemize}
	\item \gls{sap} \gls{erp} 6.0 (Kapitel \ref{sec:erp-definition})
	\item \gls{sap} \gls{crm} 7.0 (Kapitel \ref{sec:crm-definition})
	\item \gls{sap} \gls{srm} 7.0 (Kapitel \ref{sec:srm-definition})
	\item \gls{sap} \gls{scm} 7.0 (Kapitel \ref{sec:scm-definition})
	\item \gls{sap} \gls{plm} 7.0 (Kapitel \ref{sec:plm-definition})
\end{itemize}

\cite{SAPNews}

\section{Small and Medium Enterprises}
%%%%%%%%%%%%%%%%%%%%%%
%% KAPITEL ALLINONE %%
%%%%%%%%%%%%%%%%%%%%%%
%% JONAS
%%%%%%%%%%%%%%%%%%%%%%
\subsection{SAP All-in-One}
\label{sec:allinone}

Die \gls{sap} All-in-One Lösung bietet ein \gls{sap} \gls{erp} und \gls{sap} \gls{nw} für mittelständische Unternehmen. Ein Basissystem ist schon ab 90.000 Euro erhältlich und lässt sich nach den Wünschen der Kunden skalieren.

All-in-One basiert auf vordefinierten, branchenspezifischen Geschäftsprozessen. Diese wurden mit der langjährige Erfahrung der \gls{sap} im Bereich Unternehmenssoftware entwickelt. Dadurch lassen sich All-in-One Systeme schnell aufsetzen und erzeugen keine unnötigen Kosten. Der Kunde muss trotzdem nicht auf Flexibilität verzichten, da die Geschäftsprozesse genau an die Bedürfnisse der Firma angepasst werden können.

\gls{sap} All-in-One kann durch spezifische Lösungen erweitert und noch spezieller auf das eigene Unternehmen zugeschnitten werden.

Branchenlösungen sind vorhanden für Automobilzulieferer, Komponentenfertiger, Kleinserienfertiger, Kunststoffverarbeiter und Metallverarbeiter \cite{AiOBeratung}.

All-in-One ist gedacht um die Kernprozesse des Unternehmens zu automatisieren und so die Innovations- und Wachstutmsfähigkeit des Unternehmens zu erhöhen.

%%%%%%%%%%%%%%%%%
%% KAPITEL ByD %%
%%%%%%%%%%%%%%%%%
%% JONAS
%%%%%%%%%%%%%%%%%
\subsection{SAP Business By Design}
\label{sec:byd}

\gls{sap} \gls{byd} ist eine \gls{erp} \gls{ondemand} Cloudlösung für \gls{sme} ab 25 Mitarbeitern. Die Nutzung ist preiswert und skalierbar, da auf monatlicher Basis bezahlt wird und Nutzerlizenzen dynamisch hinzugekauft werden können. Die Software wird schnell bereitgestellt und der Kunde hat keine weiteren IT-Aufwendungen, da das System bei \gls{sap} direkt im Rechenzentrum gehostet wird.

\gls{byd} enthält dabei alle nötigen vorkonfigurierten Workflowprozesse, von Verwaltung der Kundenbeziehungen, Materialbeschaffung und Lieferkettenverwaltung, bis hin zu Rechnungswesen und Werbeplanung. Trotzdem verliert der Kunde kaum Flexibilität gegenüber den etablierten \gls{sap}-\gls{erp} Lösungen, wie \gls{zb} \gls{sap} Business One (siehe \ref{sec:business-one}), da der Lösungsumfang konfiguriert werden kann, um ein möglichst breites Spektrum an Aufgaben abdecken zu können. Jedoch bietet \gls{byd} kein eigentliches Customizing \cite{ERP4Students}, da die einzelnen Geschäftsprozesse nur noch geringfügig den Bedürfnissen der Firma angepasst werden können.

%%%%%%%%%%%%%%%%%%%%%%%%%%
%% KAPITEL Business One %%
%%%%%%%%%%%%%%%%%%%%%%%%%%
%% JONAS
%%%%%%%%%%%%%%%%%%%%%%%%%%
\subsection{SAP Business One}
\label{sec:business-one}

Business One ist die dritte \gls{sap}-Lösungen für \gls{sme}. Sie wird im \gls{ondemand}- oder Vor-Ort-Modell unterstützt. Stellt also eine Art Mittelweg zwischen All-in-One(\ref{sec:allinone}) und \gls{byd} dar. Wenn ein schneller Datenzugriff bereitgestellt werden muss läuft \gls{sap} Business One auch auf der In-Memory-Computing-Plattform \gls{sap} HANA.

\gls{sap} und seine Partner stellen für Business One über 550 Branchenlösungen mit vorkonfigurierten Workflows bereit. Somit kauft der Kunde eine Lösung, die schon von vielen Unternehmen genutzt wird. Dadurch werden natürlich Kosten und Risiken gesenkt, da mögliche Probleme bereits vorher aufgetreten sind und somit schnell und kostengünstig gelöst werden können.

Natürlich sind auch hier alle Workflows konfigurierbar und können über unternehmensspezifisches Customizing in nur 2 - 8 Wochen auf den Kunden zugeschnitten werden \cite{BusinessOne}.

In Business One können alle Prozesse eines Unternehmens abgebildet werden und die Mitarbeiter haben sogar externen Zugriff auf das System via \gls{sap} mobile Apps.

%%%%%%%%%%%%%%%%%%%%%%%
%% KAPITEL VERGLEICH %%
%%%%%%%%%%%%%%%%%%%%%%%
%% STEFFEN
%%%%%%%%%%%%%%%%%%%%%%%
\subsection{Vergleich der Produkte}
\begin{table}[H]
\begin{center}
\begin{tabular}{p{3.8cm}||p{3cm}|p{3cm}|p{3cm}}
  \emph{\gls{sap} \gls{sme} Lösung} & \emph{\gls{sap} Business One (\ref{sec:business-one})} & \emph{\gls{sap} \gls{byd} (\ref{sec:byd})} & \emph{\gls{sap} All-In-One (\ref{sec:allinone})}\\	
  \hline
  kurze Beschreibung & Eine einzelne, integrierte Anwendung mit der man ein gesamtes Unternehmen verwalten kann & Die Beste \gls{ondemand} Lösung von SAP & Umfassende, integrierte und sehr einfach als \gls{saas} konfiguriert\\
  \hline
  Anzahl der Nutzer & bis zu 100 & 100 bis 500 &  bis zu 2.500\\
  \hline
  Länderverfügbarkeit & 40 Länder & US, UK, D, F, Indien, China & 50 Länder\\
  \hline	
  Implementierungsart & \gls{onpremise} & \gls{ondemand} & \gls{onpremise} oder Hosted\\
  \hline	
  Implementierungszeit & 2-8 Wochen & 4-8 Wochen & 8-16 Wochen\\
  \hline	
  Transaktionsvolumen & niedrig & mittel & hoch\\
  \hline	
  Industrielösungen & mehrere & wenige & viele\\
  \hline				
\end{tabular}
\end{center}
% Beschriftung festlegen:
\caption{Vergleich der \gls{sap} \gls{sme} Produkte} 
% ein Label definieren, mit dessen Hilfe man (an beliebiger Stelle im Dokument) Bezug nehmen kann:
\label{tab:smevergleich}
\end{table}

Tabelle \ref{tab:smevergleich} zeigt einen Vergleich zwischen den verschiedenen Produkten, \gls{sap} Business One (\ref{sec:business-one}), \gls{sap} \gls{byd} (\ref{sec:byd}) und zum Schluss noch \gls{sap} All-In-One (\ref{sec:allinone}). Neben einer kurzen Beschreibung zu dem Produkt finden sich in dieser Tabelle auch die geeigneten Nutzer- bzw. Mitarbeiterzahlen, die Länderverfügbarkeit und andere Vergleiche wie die Implementierungszeit. Hier erkennt man auch wieder, wie verschieden die Produkte doch sind, was viele potentielle Kunden nicht unbedingt gleich vermuten. So ist die \gls{sap} \gls{byd}-Lösung zum Beispiel nur in sechs Ländern verfügbar, wohingegen die anderen beiden in 40 und in 50 Ländern verfügbar sind \cite{SAPin24hrs}.


\chapter{SAP Basis}  \label{chap:sap-basis}
%%%%%%%%%%%%%%%%%%%%
%% KAPITEL Server %%
%%%%%%%%%%%%%%%%%%%%
%% JONAS (Hilfe: Steffen)
%%%%%%%%%%%%%%%%%%%%%
\section{Server}

\subsection{Applikationsserver}
\label{sec:app-server}

\subsection{Storageserver}
\label{sec:stor-server}

\subsection{Betriebssysteme}
\label{sec:server-os}

\section{SAP NetWeaver Plattform}
\label{sec:netweaver}

%%%%%%%%%%%%%%%%%%%%%%%
%% KAPITEL Datenbank %%
%%%%%%%%%%%%%%%%%%%%%%%
%% STEFFEN
%%%%%%%%%%%%%%%%%%%%%%%
\section{Datenbank}

\subsection{SAP HANA}
\label{sec:db-hana}

\subsubsection{Einführung}
\label{sec:db-hana-intro}
% historische, hana studio, rowstore (anderer Aufbau als bei herkömml. dbs)
\gls{sap} \gls{hana} kombiniert die Funktionen einer \gls{db}, der Datenverarbeitung und die Funktionen einer Anwendungsplattform auf Ebene des Hardware Arbeitsspeichers. \gls{hana} bietet \gls{ua} Bibliotheken für Vorhersage, Planung, Textanalyse oder Geschäftsanalysen an.\\

\begin{figure}[H]
	\begin{center}
	\includegraphics[width=1\linewidth]{grafiken/hana-features-overview.png}
	\vspace{-20pt}
	\caption{Aufbau der \gls{sap} \gls{hana} Plattform \cite{SAPHanaAbout}}
	\vspace{-10pt}
	\label{abb:SAPHanaAbout}
	\end{center}
\end{figure}

\gls{hana} verwendet in seiner \gls{db} einen sogenannten spaltenbasierten Datenspeicher, welcher im Arbeitsspeicher abgespeichert wird. Dieser Datenspeicher ist durch verschiedene Sicherheitsfeatures vor Datenverlust bei Stromausfall oder ähnlichem gesichert.
Dadurch, dass Anwendungen direkt auf der \gls{hana} Instanz ausgeführt werden können, vereinfacht es die Entwicklung von Applikationen im Umfeld von großen Datenquellen und Datenstrukturen. In Abbildung \ref{abb:SAPHanaAbout} ist die Struktur von \gls{hana} abgebildet.

\subsubsection{Hands On}
\label{sec:db-hana-ho}
% welche wichtigen Befehle gibt es

\subsubsection{Vergleich}
\label{sec:db-hana-vgl}
% zeitvergleich oracle / hana db select

\subsection{Sonstige}
\label{sec:db-sonstige}
% welche datenbanken kann man sonst benutzen
% oracle,...

\chapter{SAP Workflow Builder}  \label{chap:builder}
%%%%%%%%%%%%%%%%%%%
%% KAPITEL Intro %%
%%%%%%%%%%%%%%%%%%%
\section{Einführung}
% MARCO
% was ist es, wer benutzt es, wofür braucht man es
% erreichbar durch Transaktion SWDD, im SAP System eingebaut usw
%%%%%%%%%%%%%%%%%%%%%%%%%%%%%%%%%%%%%%%%%%%%%%%%%%%%%%%%%%%%%%%%%
% http://www.connexin.net/de/sap-transaktionen-uebersicht.html
% -> Hier auch die Workflow transaktionen mit LOG etc erwähnen!
% http://help.sap.com/saphelp_nw04s/helpdata/EN/c5/e4b79d453d11d189430000e829fbbd/content.htm
% -> Bild zum Beispiel für die Übersicht des Builders zu erklären
% http://www.sdn.sap.com/irj/scn/go/portal/prtroot/docs/library/uuid/3c2b9c90-0201-0010-ab86-a574c7881607?QuickLink=index&overridelayout=true&5003637211723
% -> schöne Definitionen und Übersicht des Builders
% http://www.edv-buchversand.de/sap/chapter.php?cnt=getchapter&id=gp-9285.pdf
% -> auch sehr gut für Übersicht!!
% http://www.abap-tutorials.com/wp-content/uploads/pdfs/workflow_tutorial.pdf
% -> geil für die Einführung mit Warum brauchen wir überhaupt Workflows etc..
\subsection{Warum ein SAP Workflow Builder?}
\label{sec:warum-wf-builder}
Durch eine sehr breite Produktpalette und lange Erfahrung ist in einem \gls{sap} System standardmäßig eine sehr große Menge an Arbeitsabläufen vorhanden und direkt einsetzbar. Aufgrund der Verschiedenheit individueller Firmen und Branchen ist es allerdings unmöglich, alle möglichen Workflows zu integrieren und zur Verfügung zu stellen. Daher stellt die \gls{sap} ihren Kunden eine Möglichkeit zur Verfügung, mit der sie, nach einer gewissen Einarbeitungszeit, beliebige Workflows selbst abbilden können. Dadurch können gekaufte Produkte mit einer maximalen Genauigkeit in die vorhandenen Betriebsabläufe zu integriert und auch schon vorhandene Fremdsysteme angesprochen werden \cite{SAPHelpWf}.

\subsubsection{Vorteile des SAP Workflow Builders}
\label{sec:vorteile-sap-wf-builder}
Durch die direkte Einbindung in das \gls{sap} System hat der Workflow Builder einige Möglichkeiten und Funktionen, die mit einem externen Programm nicht umsetzbar wären. So ist es möglich, auf interne Ereignisse zu warten und auf diese zu reagieren. Des weiteren können auch globale Ereignisse ausgelöst werden und es kann problemlos mit anderen Transaktionen des Systems zusammengearbeitet werden. 

Da viele Firmen zur Verwaltung der Produktion, des Personals und anderen Dingen größtenteils \gls{sap} Systeme im Einsatz haben, ist es somit möglich, ein Maximum an Automatisierung zu erreichen.

\subsection{Programmoberfläche}
\label{sec:win-overwiev}
Die Programmoberfläche des Workflow Builders (siehe \ref{abb:workflow-overview}) ist in verschiedene Bereiche unterteilt. Die wichtigsten sind die im Folgenden beschriebenen.

\begin{figure}[h]
	\begin{center}
	\includegraphics[width=1.0\textwidth]{grafiken/wf-builder_overview.png}
	\caption{Programmübersicht: Der SAP Workflow Builder}
	\vspace{-10pt}
	\label{abb:workflow-overview}
	\end{center}
\end{figure}

\subsubsection{Workflow}
\label{sec:win-overview-wf}
Dieser Bereich ist der wichtigste und größte. Hier wird der Bereich des modellierten Arbeitsablaufs, der gerade bearbeitet wird, groß dargestellt und es können neue Schritte eingefügt werden, vorhandene Schritte editiert und gelöscht werden. Ein Doppelklick auf einen Schritt bringt den Benutzer zur gespeicherten Definition des Elements, welche dort gepflegt werden kann.

\subsubsection{Übersicht}
\label{sec:win-overview-uebersicht}
Die grafische Übersicht bietet dem Bearbeiter stets einen Überblick des gesamten Workflows, wofür dieser bei großen Modellierungen stark verkleinert dargestellt werden muss. Zusätzlich signalisiert ein grüner rechteckiger Rahmen stets, welcher Teil des Gesamtbildes aktuell im großen Workflow Fenster bearbeitet wird. Durch Verschieben des Rahmens ist es möglich, direkt zu einem gewünschten Teil zu springen.

\subsubsection{Schritttypen}
\label{sec:win-overview-schrittypen}
Der untere linke Bereich des Programms hat standardmäßig den Titel "Einfügbare Schrittypen" und enthält eine Liste aller Schrittypen, die verwendet werden können. Von hier können diese mit der Maus per \gls{dragdrop} in den Prozess eingefügt werden. Beim Einfügen des Schrittes wird durch ein kleines Plus am Mauszeiger signalisiert, dass der entsprechende Schritt an dieser Stelle eingefügt werden kann.

\subsubsection{Informationsbereich}
\label{sec:win-overview-information}
Der Informationsbereich zeigt an, welcher Workflow aktuell geladen ist, dessen Status und Versionsnummer. Durch einen Klick auf die Auswahlliste neben "Version" kann eine andere Version des gespeicherten Prozesses geladen werden. Um einen neuen Prozess zu laden, kann entweder, wenn diese bekannt ist, die entsprechende Identifikationsnummer in das Textfeld neben "Workflow" eingegeben werden oder die Suchhilfe mittels des kleinen Pfeils daneben geöffnet werden. Letzteres öffnet das in Abbildung \ref{abb:workflow-search} gezeigte Fenster, in welchem die auf dem System vorhandenen Workflows nach Kategorien aufgegliedert angezeigt werden.

\begin{figure}[h]
	\begin{center}
	\includegraphics[width=300px]{grafiken/wf-builder_search.png}
	\caption{Die Suchhilfe des Workflow Builders}
	\vspace{-10pt}
	\label{abb:workflow-search}
	\end{center}
\end{figure}

\subsubsection{Navigationsbereich}
\label{sec:win-overview-information}
Der Navigationsbereich beinhaltet eine Liste aller im Prozess vorhandenen Schritte. Von hier aus ist es möglich, direkt zu der Definition eines gewünschten Schrittes zu springen. 

\subsubsection{Meldungen}
\label{sec:win-overview-meldungen}
In diesem Bereich werden Nachrichten zur Information des Benutzers angezeigt. Dies können allgemeine Benachrichtigungen, Ergebnisse der Syntaxprüfung und Suchergebnisse sein.

\subsubsection{Alternative Inhalte}
\label{sec:win-overview-alternative}
Zusätzlich zu den standardmäßig beim Programmstart und in Abbildung \ref{abb:workflow-overview} angezeigten Informationen kann die Ansicht \nameref{sec:win-overview-schrittypen} zu einer alternativen Ansicht geändert werden. Dies erfolgt, indem der Benutzer auf die Überschrift "Einfügbare Schritttypen" des Bereichs klickt. Aus dem nun geöffneten Menü (siehe \ref{abb:workflow-alternatives}) ist einer der Einträge auszuwählen. Die folgenden Ansichten stehen zur Verfügung:

\begin{figure}[h]
	\begin{center}
	\includegraphics[width=150px]{grafiken/wf-builder_alternative-inhalte.png}
	\caption{Alternative Anzeigemöglichkeiten des Workflow Builders}
	\vspace{-10pt}
	\label{abb:workflow-alternatives}
	\end{center}
\end{figure}

\begin{enumerate}
	\item Der \textbf{Workflow Container} beinhaltet alle Elemente, wie Variablen und Benutzereingaben, welche während der Ausführung des Workflows benötigt werden. Neben den automatisch generierten Container Elementen können auch vom Benutzer definierte Elemente angelegt werden.
	\item Die Ansicht \textbf{Meine Workflows und Aufgaben} bietet einen Schnellzugriff auf alle Workflows, die in letzter Zeit bearbeitet wurden. Des weiteren kann eine eigene Liste an Aufgaben und Workflows angelegt werden.
	\item \textbf{Dokumentvorlagen} sind Dokumente externer Programme (Excel-Tabellen, Word-Dateien oder beliebige andere), welche im Schritt "Dokument aus Vorlage" eingebunden werden können. 
	\item \textbf{Workflow Wizards} bieten dem Benutzer die Möglichkeit, häufig genutzte Prozessteile mit Hilfe eines von \gls{sap} bereitgestellten \gls{wizard}s einzufügen.
	\item In der Ansicht \textbf{Teamworking} kann nach Schritten gesucht werden, welche von einer bestimmten Person als letztes bearbeitet wurden.
	\item Der Punkt \textbf{Workflows dieser Definition (Ausgang)} zeigt alle zur Zeit auf dem System ausgeführten Instanzen dieser Workflow Version.
	\item Der letzte Punkt, \textbf{Note it!} bietet dem Benutzer die Möglichkeit, sich Notizen zu seiner aktuellen Arbeit zu erstellen.
\end{enumerate}



\subsection{Funktionen des Builders}
\label{sec:builder-funktionen}
% MARCO
% welche Funktionalitäten hat der Builder...
Im Folgenden sollen nun zuerst die wichtigsten Funktionen des \gls{sap} Workflow Builders erklärt werden. Danach folgt im Kapitel \nameref{sec:builder-elemente} eine breiter gefächerte tabellarische Übersicht. Dort sind auch die Symbole der Schrittypen mit aufgeführt. 

Beim ersten Start des Programms wird dem Benutzer statt einer leeren Arbeitsfläche der minimale Aufbau eines Workflows im \gls{sap}-System angezeigt. (Siehe hierzu Abbildung \ref{abb:workflow-easy}) Dieser besteht aus dem Startereignis "`Workflow gestartet"' und dem Endereignis "`Workflow beendet"'. Dazwischen können beliebige Schritte an Stelle des unbekannten Schrittes (gekennzeichnet durch einen Pfeil auf weißem Hintergrund) eingefügt werden.

\begin{figure}[h]
	\begin{center}
	\includegraphics[width=150px]{grafiken/wf-builder_new-wf.png}
	\caption{Der initiale Workflow des Builders}
	\vspace{-10pt}
	\label{abb:workflow-easy}
	\end{center}
\end{figure}

\subsubsection{Aktivität}
Der wichtigste Schritttyp ist die Aktivität, welche verschiedene Aufgaben erfüllen kann. Der Benutzer kann entweder einen \gls{abap} \gls{objekttyp} und eine zugehörige Methode oder eine im System vorhandene und schon definierte Aufgabe auswählen. Die entsprechende Aktivität wird dann vom System automatisch gestartet, wenn die Stelle im laufenden Workflow erreicht wird \cite{SAPHelpWf}.

\subsubsection{Web-Aktivität}
Mit Hilfe dieses Schrittes wird aus dem internen Workflow heraus ein XML-Dokument an eine URL gesendet. Der Empfänger kann beispielsweise ein anderes System sein, welches daraufhin einen eigenen Workflow startet. Alle \gls{sap}-Systeme stellen einen Service zur Verfügung, welcher in diesem Fall automatisch einen weiteren Workflow starten kann \cite{SAPHelpWf}.

\subsubsection{Mail versenden}
Dieser Schritt versendet eine Nachricht innerhalb des \gls{sap}-Systems. Der Empfänger (es sind mehrere Empfänger möglich) kann diese im internen Postfach abrufen. Der Text der Mail wird bei der Definition des Schrittes festgelegt, wobei Variablen verwendet werden können, welche zur Laufzeit mit den entsprechenden Werten gefüllt werden \cite{SAPHelpWf}.

\subsubsection{Formular}
Ein Formular kann innerhalb des Workflows zur Anzeige von Daten oder deren Bearbeitung durch den Endnutzer verwendet werden. Nachdem bei der Definition des Schrittes die zu bearbeitenden Daten angegeben wurden, erzeugt das Workflow-System automatisch das zugehörige Formular, welches noch bearbeitet werden kann \cite{SAPHelpWf}. 

\subsubsection{Benutzerentscheidung}
Eine Benutzerentscheidung kann mit einem Text versehen werden, welcher dem Endnutzer erklärt, welche Entscheidung er treffen muss. Der Workflow kann so konfiguriert werden, dass er, je nachdem welche der vorgegebenen Antwortmöglichkeiten ausgewählt wurde, einen anderen Pfad wählt \cite{SAPHelpWf}.

\subsubsection{Bedingungen}
Die Schritte Bedingung und Mehrfachbedingung bestimmen, ähnlich der Benutzerentscheidung, den weiteren Ablauf des Workflows. Der Unterschied besteht darin, dass das System die Entscheidung eigenständig nach vorgegebenen Bedingungen fällt und der Benutzer keinen Einfluss darauf hat \cite{SAPHelpWf}.

\subsubsection{Schleifen}
Die WHILE- und UNTIL-Schleifen können eingesetzt werden, wenn ein bestimmter Teil des Workflows ausgeführt werden soll, während eine bestimmte Bedingung wahr ist oder so lange, bis sie eintritt. Schleifen können sämtliche Schrittypen (auch weitere Schleifen) enthalten und sorgen dafür, dass ein Workflow übersichtlich bleibt \cite{SAPHelpWf}.

\subsection{Schritttypen}
\label{sec:builder-elemente}
% MARCO, STEFFEN
% tabelle mit elementen..
% was ist wofür gedacht
	\begin{longtable}{|c|p{2.2cm}|p{10.8cm}|}
		\hline
		\textbf{Symbol} & \textbf{Schritttyp} & \textbf{Beschreibung}\\
		\hline
		\includegraphicstotab[width=0.8cm]{grafiken/aktivitaet.png}
		& 
		Aktivität & Ausführen einer ABAP-Methode oder einer vordefinierten Aufgabe \\ 
		\hline \includegraphicstotab[width=0.8cm]{grafiken/web-aktivitaet.png} 
		& 
		Web-Aktivität & \gls{xml}-Dokument an eine URL senden, z.B. um Workflows in Fremdsystemen zu starten\\ 
		\hline 
		\includegraphicstotab[width=0.8cm]{grafiken/mail-versenden.png} 
		& 
		Mail-Versendung & Nachricht an Endnutzer versenden\\ 
		\hline 
		\includegraphicstotab[width=0.8cm]{grafiken/formular.png}
		& 
		Formular-schritt & Anzeige von Daten und Möglichkeit zum Bearbeiten dieser durch Endnutzer\\ 
		\hline 
		\includegraphicstotab[width=0.8cm]{grafiken/benutzerentscheidung.png}
		& 
		Benutzer-entscheidung & Beantworten einer Frage bzw. Treffen einer Entscheidung durch den Benutzer zur Beeinflussung des Workflows\\ 
		\hline 
		\includegraphicstotab[width=0.8cm]{grafiken/dokument-aus-vorlage.png}
		& 
		Dokument aus Vorlage & Anzeigen oder Bearbeiten von Dokumenten, die mit externen Anwendungen erstellt wurden mit Hilfe eines auf dem Rechner installierten Programms\\ 
		\hline 
		\includegraphicstotab[width=0.8cm]{grafiken/bedingung.png}
		& 
		Bedingung & Bedingte, selbstständige Entscheidung für einen Pfad aus zwei Möglichkeiten durch das System\\ 
		\hline 
		\includegraphicstotab[width=0.8cm]{grafiken/mehrfachbedingung.png}
		& 
		Mehrfach-bedingung & Bedingte, selbstständige Entscheidung für einen Pfad aus mehreren Möglichkeiten durch das System\\ 
		\hline 
		\includegraphicstotab[width=0.8cm]{grafiken/ereigniserzeuger.png}
		& 
		Ereignis-erzeuger & Auslösen eines Ereignisses, auf welches ein Warteschritt wartet\\ 
		\hline 
		\includegraphicstotab[width=0.8cm]{grafiken/warten.png}
		& 
		Warteschritt & Warten, bis ein durch einen Ereigniserzeuger generiertes Ereignis eintritt\\ 
		\hline 
		\includegraphicstotab[width=0.8cm]{grafiken/containeroperationen.png}
		& 
		Container-operationen & Verändern von Elementen des Workflow-Containers (Umgebung des aktiven Workflows mit Variablen und Benutzerentscheidungen)\\ 
		\hline 
		\includegraphicstotab[width=0.8cm]{grafiken/ablaufsteuerung.png}
		& 
		Ablauf-steuerung & Eingriff in den Ablauf des aktuellen Workflows - Abbruch oder Beenden einzelner Schritte oder des gesamten Workflows\\ 
		\hline 
		\includegraphicstotab[width=0.8cm]{grafiken/schleife.png}
		& 
		Schleifen & Mehrfache Ausführung eines Blocks von Schritten unter einer bestimmten Bedingung\\ 
		\hline 
		\includegraphicstotab[width=0.8cm]{grafiken/paralleler-abschnitt.png}
		& 
		Paralleler Abschnitt & Aufsplitten des Workflows in zwei parallel laufende Pfade\\ 
		\hline 
		\includegraphicstotab[width=0.8cm]{grafiken/ad-hoc-anker.png}
		& 
		Ad-hoc-Anker & Möglichkeit, einen anderen Workflow des Systems zu hinterlegen, der vom berechtigten Benutzer ausgeführt werden kann\\ 
		\hline 
		\includegraphicstotab[width=0.8cm]{grafiken/block.png}
		& 
		Block & Zusammenfassen mehrerer Schritte zu einem Block mit eigenen Variablen\\ 
		\hline 
		\includegraphicstotab[width=0.8cm]{grafiken/lokaler-workflow.png}
		& 
		Lokaler Workflow & Einfügen eines Sub-Workflows, welcher vollen Zugriff auf die Daten des aktuellen Workflows hat\\ 
		\hline
	% Beschriftung festlegen:
	\caption{Symbolerklärung des \gls{sap} Workflow Builders}
	% ein Label definieren, mit dessen Hilfe man (an beliebiger Stelle im Dokument) Bezug nehmen kann:
	\label{tab:builderelemente}		
	\end{longtable} 



%%%%%%%%%%%%%%%%%%%%%
%% KAPITEL HandsOn %%
%%%%%%%%%%%%%%%%%%%%%
%% MARCO
%%%%%%%%%%%%%%%%%%%%%
\section{Hands On}

\subsection{Erster Beispielworkflow}
\label{sec:builder-1-bsp}
% kleiner sinnloser workflow (schleife,...) => aus dem Video Tutorial

\subsection{Zweiter Beispielworkflow}
\label{sec:builder-2-bsp}
% demo workflow aus den vorlagen nehmen (abwesenheitsbestätigung)

\subsubsection{Vorstellung des Workflows}
\label{sec:builder-2-bsp-vorstellung}
% wofür ist der workflow gut, was soll er tun (aus anwendersicht)

\subsubsection{Umsetzung des Workflows}
\label{sec:builder-2-bsp-umsetzung}
% technische sicht, "`klickbares"' howto

%%%%%%%%%%%%%%%%%%%%%%%%%%
%% KAPITEL Fremdsysteme %%
%%%%%%%%%%%%%%%%%%%%%%%%%%
%% JONAS 
%%%%%%%%%%%%%%%%%%%%%%%%%%
\section{Schnittstellen}

\subsection{SAP Fremdsysteme}
\label{sec:export-sap}

\gls{sap} Systeme liefern Workflows, die auf das Ziel der Applikation ausgelegt sind. \gls{erp}, \gls{crm} und \gls{srm} sind Beispiele für Systeme, die eingebaute, vordefinierte Workflows bereitstellen. 

Die Workflows sind anpassbar, um den Bedürfnissen der Firma gerecht zu werden. Es können mit dem Workflowbuilder ganz eigene Geschäftsprozesse entwickelt werden, die natürlich über Modulgrenzen hinweg Zugriff auf Daten besitzen. So können Daten aus einem \gls{crm}-System in einem \gls{erp} zur Analyse, Auswertung und Bearbeitung von Daten hinzugezogen werden.

\subsection{XML}
\label{sec:export-xml}
% was ist dieses Format
% in welche Programme kann man es importieren?

\gls{xml} ist die Abkürzung für E\textbf{x}tensible \textbf{M}arkup \textbf{L}anguage und bezeichnet eine Auszeichnungssprache. Mit dieser können hierarchisch strukturierte Daten in Textform dargestellt werden. \gls{xml} besteht aus Elementen, deren Name, bis auf ein paar Ausnahmen, frei gewählt werden darf. Elemente haben einen Anfangs- ($\langle$elementName$\rangle$) und einen Endtag ($\langle$/elementName$\rangle$). Zwischen den Tags können weiter Elemente, Text und Knoten stehen. Diese sind dem Element dann untergeordnet.

Das World Wide Web Consortium, kurz \gls{w3c}, hat \gls{xml} als eine Metasprache definiert, auf deren Basis anwendungsspezifische Auszeichnungssprachen entwickelt werden können. Diese werden beschrieben durch ein Schema, welches festlegt, welche Elemente verwendet werden dürfen und welches Verhalten diese aufweisen \cite{XML}. So ist \gls{zb} auch XHTML definiert.

\subsection{BPMN und BPML}
\label{sec:export-bpmn-bpml}
% was ist dieses Format
% in welche Programme kann man es importieren?

\textbf{B}usiness \textbf{P}rocess \textbf{M}odel and \textbf{N}otation (\gls{bpmn}) ist eine grafische Spezifikationssprache, welche Symbole bereitstellt mit deren Hilfe Geschäftsprozesse und Arbeitsabläufe dargestellt werden können.\cite{BPMN} \gls{bpmn} wurde 2005 von der \gls{omg}, auch zuständig für \gls{zb} \gls{uml}, übernommen und gewann ab dann an Bedeutung in der Informatik. Außerdem wurde sie 2013 zum internationalen Standard (ISO/IEC 19510:2013) erhoben \cite{OMG}.

Da sich \gls{bpmn} rein auf die Darstellung von Workflows bezieht wurden mehrere, von \gls{xml} abgeleitete, Auszeichnungssprachen entwickelt, um Business Process Models auch als, für einen Computer verständliche, Daten aufschreiben zu können. Dazu zählen \gls{zb} \gls{bpel}, \gls{xpdl} oder \gls{bpml} \cite{BPMN}.

Die \textbf{B}usiness \textbf{P}rocess \textbf{M}odeling \textbf{L}anguage (\gls{bpml}) wird von \gls{sap} im Workflowbuilder (\ref{chap:builder}) verwendet um Geschäftsprozesse zu exportieren. Da \gls{bpml} auch unter dem Dach der \gls{omg} steht wird sie auch in anderen Workflow Management Systemen, wie \gls{zb} jBPM, Camunda BMP oder ARIS, verwendet. Dadurch lassen sich \gls{sap}-interne Geschäftsprozesse auch extern einbetten \cite{BPML}.

\chapter{SAP Business By Design}  \label{chap:byd}
%%%%%%%%%%%%%%%%%%%
%% KAPITEL Intro %%
%%%%%%%%%%%%%%%%%%%
%% JONAS         %%
%%%%%%%%%%%%%%%%%%%
\section{Einführung}

\gls{sap} \gls{byd} ist eine \gls{erp} \gls{ondemand} Cloudlösung. Die Nutzung wird monatlich bezahlt. Dadurch können Nutzerlizenzen dynamisch erworben werden und der Kunde bezahlt immer nur so viel, wie er muss.

\gls{byd} ist preiswert und skalierbar. Die Software wird innerhalb weniger Wochen bereitgestellt. Außerdem wird das System direkt bei \gls{sap} vor Ort im Rechenzentrum gehostet, sodass der Kunde keine weiteren IT-Investitionen tätigen muss.

\gls{byd} enthält alle nötigen vorkonfigurierten Geschäftsprozesse, von Verwaltung der Kundenbeziehungen, Materialbeschaffung und Lieferkettenverwaltung, bis hin zu Rechnungswesen und Werbeplanung. Trotzdem verliert der Kunde kaum Flexibilität gegenüber den etablierten \gls{sap}-Lösungen, wie \gls{zb} \gls{sap} Business One (siehe \ref{sec:business-one}).

Für Installation, Wartung und Aktualisierung der Lösung sorgt das integrierte Betriebsmodell. Alle Betriebskosten, die durch ein Vor-Ort System entstehen sind also im Preis einbegriffen. Damit kann sich der Kunde vollständig auf sein Kerngeschäft konzentrieren.

\gls{sap} \gls{byd} wird über eine sichere Internetverbindung und einen Webbrowser als dynamische Website aufgerufen. Somit können Mitarbeiter von überall auf ihren Arbeitsplatz zugreifen und müssen weder vor Ort im Büro sein oder sich anderweitig ins Firmennetz einwählen.

\subsubsection{Vorteile von ByD}

\begin{itemize}
\item Business ByDesign vereinigt alle Vorteile einer modernen Unternehmensanwendung, bei minimalen Anforderungen an die IT
\item SAP Business ByDesign greift auf bewährte Geschäftsvorfälle zu, die umgehend einsatzbereit sind
\item Der Kunde nutzt automatisch stets die aktuellste Softwareversion
\item SAP Business ByDesign schont die Investition für eine eigene IT-Infrastruktur, durch ein skalierbares Mietmodell
\item Wechselnde Geschäftsanforderungen gehen mit der Nutzung der Softwarebereiche Hand in Hand
\end{itemize}
\cite{itelligence}

%%%%%%%%%%%%%%%%%%%%%%%%%%%%%%%%
%% KAPITEL Benutzeroberfläche %%
%%%%%%%%%%%%%%%%%%%%%%%%%%%%%%%%
%% JONAS                      %%
%%%%%%%%%%%%%%%%%%%%%%%%%%%%%%%%
\section{Benutzeroberfläche}

\gls{byd} ist in verschiedene WorkCenter unterteilt, die jeweils einen bestimmten Zweck erfüllen.

\begin{figure}[H]
	\begin{center}
	\includegraphics[width=1.0\textwidth]{grafiken/ByDesign-Ubersicht.png}
	\caption{ByDesgin Übersicht}
	\vspace{-10pt}
	\label{abb:byd-overview}
	\end{center}
\end{figure}

\subsubsection{Persönliche Daten}

Hier kann der Mitarbeiter seine Daten, wie \gls{zb} Telefonnummer oder E-Mail, einstellen und sein \gls{byd} konfigurieren.

\subsubsection{Verfügbare WorkCenter}

In dieser Sektion der Anzeige kann der User die verschiedenen, für ihn verfügbaren WorkCenter auswählen. Diese werden dann unten im "Geöffnete WorkCenter" Bereich geöffnet. So kann der Benutzer zwischen mehreren WorkCentern wechseln.

\subsubsection{Arbeitsfläche}

Hier werden die eigentlichen Inhalte des Webinterfaces angezeigt. Wenn der Beispielworkflow durchgespielt wird, werden auch nur noch diese Ausschnitte des Bildschirms gezeigt.

\subsubsection{Geöffnete WorkCenter}

Im Bereich "Geöffnete WorkCenter" sieht der Mitarbeiter alle WorkCenter, die er im Moment geöffnet hat.

%%%%%%%%%%%%%%%%%%%%%%%%%%%%%%
%% KAPITEL Beispielworkflow %%
%%%%%%%%%%%%%%%%%%%%%%%%%%%%%%
%% JONAS                    %%
%%%%%%%%%%%%%%%%%%%%%%%%%%%%%%
\section{Beispielworkflow}

\subsection{Vorstellung des Workflows}
\label{sec:byd-bsp-vorstellung}
% Schulungsworkflow beschreiben (anwendersicht)

\subsubsection{Szenario}

Der Verkaufsbereichsleiter unserer Firma hat auf einer Technologiemesse ein innovatives Produkt. Er würde gerne einen neuen Solarboiler in das Produktportfolio der Firma aufnehmen. Die Nachfrage nach Innovation ist sehr groß.

\subsubsection{Aufgabe des Mitarbeiters}

\begin{enumerate}
 \item Wir müssen in einem Katalog oder elektronischen Marktplatz einen Zulieferer für das gewünschte Produkt finden.
 \item Danach müssen wir den günstigsten Zulieferer finden, der gleichzeitig auch eine hohe Verfügbarkeit gewährleisten kann.
 \item Wenn wir ein passendes Produkt gefunden haben müssen wir dieses in \gls{sap} \gls{byd} einfügen und ihm eine Produktkategorie zuordnen.
 \item Währenddessen müssen alle wichtigen Daten über das Produkt in das System eingepflegt werden.
 \item Als Letztes müssen wir den Zulieferer für das neue Material im \gls{byd} einfügen.
 \end{enumerate}

\subsection{Umsetzung des Workflows}
\label{sec:byd-bsp-umsetzung}
% technische sicht, "`klickbares"' howto

\subsubsection{Produktsuche}

Im ersten Schritt suchen wir uns ein Produkt und einen Zulieferer auf der Website \url{alibaba.com}. Auf dieser Website können kleine Unternehmer ihre Produkte zum Verkauf anbieten. Im Moment sind über 2 Millionen Zulieferer registriert.

Für unser Beispiel verwenden wir den \href{http://www.alibaba.com/product-detail/SunSurf-SC-IP01-solar-boiler-system_627442099.html?s=p}{SunSurf SC-IP01} Solar Boiler. Dieser kostet zwischen 400 und 500 USD und muss mindestens zu 15 Stück bestellt werden. Der Zulieferer kann maximal 5000 Stück im Monat liefern.

\subsubsection{Produkt im System anlegen}

Im WorkCenter "Produktportfolio" können wir nun die Daten des SunSurf SC-IP01 unter einem neuen Material abspeichern. Dazu klicken wir auf "Produkte nach Materialien" und dann auf "Neu". In diesem Formular geben wir nun die Daten den Solar Boilers an:

\begin{figure}[H]
	\begin{center}
	\includegraphics[width=1.0\textwidth]{grafiken/ByDesign-HowTo-1.png}
	\caption{Neues Material anlegen}
	\vspace{-10pt}
	\label{abb:byd-newmaterial}
	\end{center}
\end{figure}

Nachdem wir auf "Sichern und schließen" geklickt haben wurde unser Material erfolgreich angelegt.

\subsubsection{Zulieferer anlegen}

Im WorkCenter "Lieferantenbasis" unter "Lieferanten" können wir nur einen neuen Zulieferer anlegen. Dazu klicken wir wieder auf "neu". In diesem Formular geben wir nun die Daten des Zulieferers ein:

\begin{figure}[H]
	\begin{center}
	\includegraphics[width=1.0\textwidth]{grafiken/ByDesign-HowTo-2.png}
	\caption{Neuen Zulieferer anlegen}
	\vspace{-10pt}
	\label{abb:byd-newsupplier}
	\end{center}
\end{figure}

Nachdem wir auf "Sichern und schließen" geklickt haben wurde unser Zulieferer erfolgreich angelegt.

%%%%%%%%%%%%%%%%%%%%%
%% KAPITEL Grenzen %%
%%%%%%%%%%%%%%%%%%%%%
%% JONAS           %%
%%%%%%%%%%%%%%%%%%%%%
\section{Grenzen von ByD}

Trotz der vielseitigen Vorteile von \gls{byd} stößt auch diese Lösung, wie alle anderen, an ihre Grenzen.

\subsubsection{Vordefinierte Geschäftsprozesse}

Durch die Idee hinter \gls{byd}, eine vorkonfigurierte On-Demand Unternehmensmanagement Applikation bereitzustellen, weißt es Nachteile gegenüber den anderen \gls{sme}-Lösungen im Bereich Customizing auf. So kann \gls{byd} nicht beliebig eingestellt werden.

\subsubsection{Module}

Da \gls{byd} in Form von Modulen zusammengestellt wird bekommt der Kunde unausweichlich auch Funktionalität, die er gar nicht benötigt und bezahlt für unnötige Anwendungsbestandteile. In diesem Aspekt sind Business One \ref{sec:business-one} oder \gls{sap} All-in-One \ref{sec:allinone} die bessere Wahl.

\subsubsection{Erweiterbarkeit}

Im Gegensatz zu den beiden anderen \gls{sme}-Systemen kann \gls{byd} nicht beliebig erweitert werden. So können nicht einfach spezifische Prozesse neu entwickelt und in das vorhandene System eingebunden werden, da \gls{byd} keine Möglichkeit bietet eigene Workflows anzulegen und auch \gls{sap} keine weiteren Add-Ons anbietet, als die Standardsoftware.



\chapter{Gesamtfazit}  \label{chap:fazit}
% irgendwie einmal ByD und Workflow Builder zusammenfassen (jeweils Jonas & Marco)
% ein kleinen Abschnitt dass ByD f�r Firmen besser ist, die mit fest definierten Prozessen einverstanden sind
% und keine eigenen brauchen, als auch f�r kleine Firmen. F�r gro�e, die viel Customizing brauchen
% ist Workflow Builder weitaus besser, da viel mehr Individualisierbar und Komplexer
% jeder Wunsch kann damit erf�llt werden. Perfekte Implementierung in vorhandene SAP Landschaft

\appendix
\chapter{Anhang}  \label{chap:anhang}
%%%%%%%%%%%%%%%%%%%%
%% KAPITEL ANHANG %%
%%%%%%%%%%%%%%%%%%%%
\section{HANA Beispieldaten}

\lstset{language=SQL, caption={Beispieldaten anlegen \cite{SAPSCN}}, label={anhang:hanasql}}								
\begin{lstlisting}
CREATE COLUMN TABLE "SALES_F" ("SALES_ORDER_NBR" BIGINT CS_FIXED NOT NULL ,
       "CALENDAR_DAY" DAYDATE CS_DAYDATE,
       "BUSINESS_UNIT_ID" BIGINT CS_FIXED,
       "MATERIAL_ID" BIGINT CS_FIXED,
       "SUPPLIER_ID" BIGINT CS_FIXED,
       "UNIT_PRICE" DOUBLE CS_DOUBLE,
       "QUANTITY_SOLD" DOUBLE CS_DOUBLE,
       PRIMARY KEY ("SALES_ORDER_NBR"));

CREATE COLUMN TABLE "BUSINESS_UNIT_D" ("BUSINESS_UNIT_ID" BIGINT CS_FIXED NOT NULL ,
       "BUSINESS_UNIT_CODE" NVARCHAR(5),
       "BUSINESS_UNIT_DESC" NVARCHAR(256),
       "PARENT_BUSINESS_UNIT_ID" BIGINT CS_FIXED,
       "PARENT_BUSINESS_UNIT_CODE" NVARCHAR(5),
       PRIMARY KEY ("BUSINESS_UNIT_ID"));

CREATE COLUMN TABLE "SUPPLIER_D" ("SUPPLIER_ID" BIGINT CS_FIXED,
       "SUPPLIER_DESC" VARCHAR(60),
       PRIMARY KEY("SUPPLIER_ID"));

CREATE COLUMN TABLE "MATERIAL_D" ("MATERIAL_ID" BIGINT CS_FIXED,
       "SKU" VARCHAR(16),
       "MATERIAL_GROUP" VARCHAR(60),
       PRIMARY KEY("MATERIAL_ID"));
 
INSERT INTO "BUSINESS_UNIT_D"
VALUES(1,'BU1','Business Unit 1',0,'');
INSERT INTO "BUSINESS_UNIT_D"
VALUES(2,'BU2','Business Unit 2',1,'BU1');
INSERT INTO "BUSINESS_UNIT_D"
VALUES(3,'BU3','Business Unit 3',1,'BU1');
INSERT INTO "BUSINESS_UNIT_D"
VALUES(4,'BU4','Business Unit 4',2,'BU2');
INSERT INTO "BUSINESS_UNIT_D"
VALUES(5,'BU5','Business Unit 5',3,'BU3');
INSERT INTO "BUSINESS_UNIT_D"
VALUES(6,'BU6','Business Unit 6',3,'BU4');
INSERT INTO "BUSINESS_UNIT_D"
VALUES(7,'BU7','Business Unit 7',4,'BU4');
INSERT INTO "BUSINESS_UNIT_D"
VALUES(8,'BU8','Business Unit 6',4,'BU4');
 
CREATE COLUMN TABLE ADJECTIVE (ID INTEGER, WORD VARCHAR(60), PRIMARY KEY ("ID"));
CREATE COLUMN TABLE NOUN (ID INTEGER, WORD VARCHAR(60), PRIMARY KEY ("ID"));
CREATE COLUMN TABLE SUP_TYPE (ID INTEGER, WORD VARCHAR(60), PRIMARY KEY ("ID"));
 
INSERT INTO ADJECTIVE VALUES(1, 'Great');
INSERT INTO ADJECTIVE VALUES(2, 'Modern');
INSERT INTO ADJECTIVE VALUES(3, 'Fast');
INSERT INTO ADJECTIVE VALUES(4, 'Proud');
INSERT INTO ADJECTIVE VALUES(5, 'Solid');
INSERT INTO ADJECTIVE VALUES(6, 'Broad');
INSERT INTO ADJECTIVE VALUES(7, 'Elegant');
INSERT INTO ADJECTIVE VALUES(8, 'Fancy');
INSERT INTO ADJECTIVE VALUES(9, 'Mysterious');
INSERT INTO ADJECTIVE VALUES(10, 'Fantastic');
 
INSERT INTO NOUN VALUES(1, 'Factory');
INSERT INTO NOUN VALUES(2, 'Offices');
INSERT INTO NOUN VALUES(3, 'Industry');
INSERT INTO NOUN VALUES(4, 'Station');
INSERT INTO NOUN VALUES(5, 'Restaurant');
INSERT INTO NOUN VALUES(6, 'Buildings');
INSERT INTO NOUN VALUES(7, 'Mall');
INSERT INTO NOUN VALUES(8, 'Studio');
INSERT INTO NOUN VALUES(9, 'Stockbrokers');
INSERT INTO NOUN VALUES(10, 'Academy');
 
INSERT INTO SUP_TYPE VALUES(1, 'Limited');
INSERT INTO SUP_TYPE VALUES(2, 'Pty Ltd');
INSERT INTO SUP_TYPE VALUES(3, 'Partnership');
INSERT INTO SUP_TYPE VALUES(4, 'Group');
INSERT INTO SUP_TYPE VALUES(5, 'Trust');
INSERT INTO SUP_TYPE VALUES(6, 'Collective');
INSERT INTO SUP_TYPE VALUES(7, 'Consortium');
INSERT INTO SUP_TYPE VALUES(8, 'Inc.');
INSERT INTO SUP_TYPE VALUES(9, 'Traders');
INSERT INTO SUP_TYPE VALUES(10, 'Franchise');
 
CREATE SEQUENCE seq START WITH 1;
 
CREATE PROCEDURE BUILD_SUPPLIER_TABLE (IN NMBR INT) LANGUAGE SQLSCRIPT AS
CNTR INTEGER;
BEGIN
CNTR := 0;
WHILE CNTR < :NMBR DO
INSERT INTO SUPPLIER_D
SELECT seq.NEXTVAL,
            (SELECT TOP 1 WORD FROM ADJECTIVE WHERE ID = SUBSTR(ROUND(RAND() * 9, 0 ),1,1) + 1 ORDER BY WORD)  || ' ' ||
            (SELECT TOP 1 WORD FROM NOUN WHERE ID = SUBSTR(ROUND(RAND() * 9, 0 ),1,1) + 1 ORDER BY WORD) ||  ' ' ||
            (SELECT TOP 1 WORD FROM SUP_TYPE WHERE ID = SUBSTR(ROUND(RAND() * 9, 0 ),1,1) + 1 ORDER BY WORD)  AS SUPDESC
FROM DUMMY;      
CNTR := CNTR + 1;
END WHILE;
END;
 
CALL BUILD_SUPPLIER_TABLE(1000);
 
CREATE COLUMN TABLE MAT_GROUP (ID INTEGER, WORD VARCHAR(60), PRIMARY KEY ("ID"));
INSERT INTO MAT_GROUP VALUES(1, 'Engine');
INSERT INTO MAT_GROUP VALUES(2, 'Exterior');
INSERT INTO MAT_GROUP VALUES(3, 'Interior');
INSERT INTO MAT_GROUP VALUES(4, 'Accesories');
INSERT INTO MAT_GROUP VALUES(5, 'Electrical');
INSERT INTO MAT_GROUP VALUES(6, 'Components');
INSERT INTO MAT_GROUP VALUES(7, 'Finishing');
INSERT INTO MAT_GROUP VALUES(8, 'Hydraulics');
INSERT INTO MAT_GROUP VALUES(9, 'Liquids');
INSERT INTO MAT_GROUP VALUES(10, 'Extras');
 
CREATE PROCEDURE BUILD_MAT_GROUP_TABLE (IN NMBR INT) LANGUAGE SQLSCRIPT AS
CNTR INTEGER;
BEGIN
CNTR := 0;
WHILE CNTR < :NMBR DO
INSERT INTO MATERIAL_D
SELECT :CNTR,
       'SKU' || LPAD(ROUND((RAND() * 1000000),0),7,'0000000') as SKU,
            (SELECT TOP 1 WORD FROM MAT_GROUP WHERE ID = SUBSTR(ROUND(RAND() * 9, 0 ),1,1) + 1 ORDER BY WORD)  AS MATERIAL
FROM DUMMY;      
CNTR := CNTR + 1;
END WHILE;
END;
 
CALL BUILD_MAT_GROUP_TABLE(10000);
 
CREATE PROCEDURE BUILD_FACT_TABLE (IN NMBR INT) LANGUAGE SQLSCRIPT AS
CNTR INTEGER;
BEGIN
CNTR := 0;
WHILE CNTR < :NMBR DO
INSERT INTO SALES_F
SELECT :CNTR,
       ADD_DAYS (TO_DATE ('2011-01-01', 'YYYY-MM-DD'), RAND() * 730),
         ROUND((RAND() * (SELECT COUNT(*) FROM BUSINESS_UNIT_D)), 0 ),
         ROUND((RAND() * (SELECT COUNT(*) FROM MATERIAL_D)), 0 ),
         ROUND((RAND() * (SELECT COUNT(*) FROM SUPPLIER_D)), 0 ),
         ROUND(RAND() * 1000,2),
         ROUND(RAND() * 100,0)
FROM DUMMY;      
CNTR := CNTR + 1;
END WHILE;
END;
 
CALL BUILD_FACT_TABLE(10000000);
\end{lstlisting}

\section{Weiterführende Screenshots zum Workflow Builder}

\begin{figure}[H]
	\begin{center}
	\includegraphics[height=0.9\textheight]{grafiken/wf-builder_view-classicepc.png}
	\caption{Ansicht eines Workflows als klassisches EPK}
	\vspace{-10pt}
	\label{abb:workflow-view-classicepc}
	\end{center}
\end{figure}

\begin{figure}[H]
	\begin{center}
	\includegraphics[height=0.9\textheight]{grafiken/wf-builder_view-epc.png}
	\caption{Ansicht eines Workflows als Mischform beider Ansichten}
	\vspace{-10pt}
	\label{abb:workflow-view-epc}
	\end{center}
\end{figure}

\begin{figure}[h]
	\begin{center}
	\includegraphics[width=250px]{grafiken/wf-builder_bsp1_formular-aufgabe_eingabehilfe-ausdruck.png}
	\caption{Eingabehilfe zum Ausdruck bei neuen Aufgaben}
	\vspace{-10pt}
	\label{abb:workflow-bsp1-aufgaben_form-inputhelp}
	\end{center}
\end{figure}

\begin{figure}[H]
	\begin{center}
	\includegraphics[width=350px]{grafiken/wf-builder_bsp1_formular-containerelement-edit.png}
	\caption{Einstellung zum Import eines Containerelements}
	\vspace{-10pt}
	\label{abb:workflow-bsp1-containeredit-import}
	\end{center}
\end{figure}

\begin{figure}[H]
	\begin{center}
	\includegraphics[height=0.7\textheight]{grafiken/wf-builder_bsp1_complete.png}
	\caption{Erster Beispielworkflow fertiggestellt}
	\vspace{-10pt}
	\label{abb:workflow-bsp1-complete}
	\end{center}
\end{figure}

\begin{figure}[H]
	\begin{center}
	\includegraphics[height=0.9\textheight]{grafiken/wf-builder_bsp2_complete.png}
	\caption{Zweiter Beispielworkflow fertiggestellt}
	\vspace{-10pt}
	\label{abb:workflow-bsp2-complete}
	\end{center}
\end{figure}

\section{Business ByDesign Screenshots}

\subsection{Material anlegen}
\label{addmat}

\begin{figure}[H]
	\begin{center}
	\includegraphics[width=1.0\textwidth]{grafiken/ByDesign-HowTo-1.png}
	\caption{Neues Material anlegen}
	\vspace{-10pt}
	\label{abb:byd-newmaterial}
	\end{center}
\end{figure}

\subsection{Zulieferer anlegen}
\label{addsup}

\begin{figure}[H]
	\begin{center}
	\includegraphics[width=1.0\textwidth]{grafiken/ByDesign-HowTo-2.png}
	\caption{Neuen Zulieferer anlegen}
	\vspace{-10pt}
	\label{abb:byd-newsupplier}
	\end{center}
\end{figure}

\subsection{Angebotsanfrage absenden}
\label{sendrfq}

\begin{figure}[H]
	\begin{center}
	\includegraphics[width=1.0\textwidth]{grafiken/ByDesign-HowTo-Ausschreibung-1.png}
	\caption{Neue Ausschreibung erstellen - Allgemeine Daten}
	\vspace{-10pt}
	\label{abb:byd-newsupplier}
	\end{center}
\end{figure}
\begin{figure}[H]
	\begin{center}
	\includegraphics[width=1.0\textwidth]{grafiken/ByDesign-HowTo-Ausschreibung-2.png}
	\caption{Neue Ausschreibung erstellen - Positionen definieren}
	\vspace{-10pt}
	\label{abb:byd-newsupplier}
	\end{center}
\end{figure}
\begin{figure}[H]
	\begin{center}
	\includegraphics[width=1.0\textwidth]{grafiken/ByDesign-HowTo-Ausschreibung-3.png}
	\caption{Neue Ausschreibung erstellen - Bieter hinzufügen}
	\vspace{-10pt}
	\label{abb:byd-newsupplier}
	\end{center}
\end{figure}
\begin{figure}[H]
	\begin{center}
	\includegraphics[width=1.0\textwidth]{grafiken/ByDesign-HowTo-Ausschreibung-4.png}
	\caption{Neues Angebot - Allgemeine Daten}
	\vspace{-10pt}
	\label{abb:byd-newsupplier}
	\end{center}
\end{figure}
\begin{figure}[H]
	\begin{center}
	\includegraphics[width=1.0\textwidth]{grafiken/ByDesign-HowTo-Ausschreibung-5.png}
	\caption{Neues Angebot - Preise einfügen}
	\vspace{-10pt}
	\label{abb:byd-newsupplier}
	\end{center}
\end{figure}
\begin{figure}[H]
	\begin{center}
	\includegraphics[width=1.0\textwidth]{grafiken/ByDesign-HowTo-Ausschreibung-6.png}
	\caption{Übersicht: Ausschreibungen}
	\vspace{-10pt}
	\label{abb:byd-newsupplier}
	\end{center}
\end{figure}

\subsection{Vertrag schließen}
\label{contract}

\begin{figure}[H]
	\begin{center}
	\includegraphics[width=1.0\textwidth]{grafiken/ByDesign-HowTo-4.png}
	\caption{Vertrag schließen1}
	\vspace{-10pt}
	\label{abb:byd-newsupplier}
	\end{center}
\end{figure}

%GLOSSARIES
\printglossary[type=\acronymtype]
%%% \newpage just to demonstrate that links are correct
\newpage
\printglossary[type=main]

% manchmal muss man etwas tricksen, damit alles auf 1 Seite passt:
% ("vertical space" mit negativem Abstand)
\vspace{-1.0cm}

%% zum guten Schluss kommt das Literaturverzeichnis
% Darstellungsart festlegen:
% http://www.cs.stir.ac.uk/~kjt/software/latex/showbst.html
\bibliographystyle{acm}

% Datei meineliteratur.bib einbinden.
\bibliography{includes/meineliteratur}

%%%%%
\end{document}
%%%%%