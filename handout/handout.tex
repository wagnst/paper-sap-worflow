\documentclass{handout}
\SetHandoutTitle{Workflowmanagement - \textbf{SAP} ERP \& \textbf{SAP} ByD}
\SetUniversitaet{Duale Hochschule Baden-Württemberg}
\SetDatum{\today}
\SetFakultaet{Angewandte Informatik IBC}
\SetSemester{Sommersemester 2014}
\SetDozent{Kai-Frank Strugalla}
\SetReferent{S. Wagner, M. Dörfer, J. Dann}
\SetMatrikelnr{8974337, 6541564, 3346893}
\SetSeminar{Proseminar Workflow (T2INF4122)}

\hypersetup{
    pdftitle={Handout \HandoutTitle},
    pdfauthor={\Referent},
    pdfsubject={},
    pdfkeywords={},
    pdfcreator={LaTeX, hyperref, KOMA-Script}, % Ersteller
  }
\begin{document}


%%% SEITE 1
%%% STEFFEN
\maketitle

\section{SAP AG}
Die \emph{SAP AG} wurde 1972 von 5 ehemaligen IBM Mitarbeitern gegründet und ist als weltweit viertgrößter Softwarehersteller \cite{SAPFacts} der Marktführer im Bereich betriebswirtschaftlicher Standardsoftware.

\small
\begin{tabular}{L}
66.500+ Mitarbeiter weltweit \cite{SAPAtGlance} \\
253.500+ Kunden in 188 Ländern \cite{SAPAtGlance} \\
16,82 Milliarden Euro Umsatz in Q4/2013 \cite{SAPFacts} \\
\end{tabular}
\normalsize

SAP hat vier Haupt-Softwareprodukte im Portfolio. Die Produkte decken die Bereiche von Enterprise Resource Planning (ERP), Supply Chain Management (SCM), Product Lifecycle Management (PLM), Supplier Relationship Management (SRM) und Customer Relationship Management (CRM) auf verschiedene Arten ab.

\small
\begin{tabular}{L ll}
SAP \emph{Business One} & $\rightarrow$ & OnPremise / OnDemand / OnDevice \\
SAP \emph{Business ByDesign} & $\rightarrow$ & OnDemand \\
SAP \emph{Business All-in-One} & $\rightarrow$ & OnPremise / OnDemand / OnDevice\\
SAP \emph{Business Suite} & $\rightarrow$ & OnPremise \\
\end{tabular}
\normalsize

\section{SAP Basis}


%%
\newpage

%%% SEITE 2
%%% MARCO
\section{SAP Workflow Builder}


%%
\newpage

%%% SEITE 3
%%% JONAS
\subsection{Schnittstellen}

Die mit dem Workflowbuilder gebauten Geschäftsprozesse bieten verschiedene Schnittstellen um mit SAP eigenen Systemen oder externen Tools zu kommunizieren:

\begin{itemize}
\item Kommunikation zwischen verschiedenen SAP-Systemen durch ABAP-Coding
\item Export in internes XML-Format
\item Export in BPML
\end{itemize}

BPML wird auch von anderen Workflow-Systemen, wie zum Beispiel jBPMN, Camuda BMP oder ARIS, verwendet.

\section{ByDesign}

SAP Business ByDesign (ByD) ist eine onDemand Cloudlösung für SME, also für kleine bis mittelständige Unternehmen. OnDemand bedeutet hier, dass ByD an den Bedarf der Firma angepasst wird. So bezahlt man monatlich je nach Anzahl der Nutzerlizenzen, die gerade im Gebrauch sind, mehr oder weniger. Das System wird bei SAP gehostet und die Installation, Wartung und Aktualisierung ist durch das integrierte Betriebsmodell durch SAP gewährleistet. Aufgerufen wird ByD über den Browser. Die Website ist auch außerhalb des Firmennetzes erreichbar, sodass User von überall arbeiten können.

\subsection{Grenzen von ByDesign}

Da ByD eine standardisierte Cloudlösung ist tun sich mehrere Probleme auf:

\begin{itemize}
\item Nur zu einem bestimmten Grad granulares Customizing möglich
\item Funktionalität in Module unterteilt (teils für den eigenen Betrieb unnötige Funktionalität)
\item Keine Erweiterbarkeit durch eigene Geschäftsprozesse oder SAP Customizing
\end{itemize}

Wenn die Kundenanforderungen die Grenzen von ByD überschreiten gibt es sicher eine andere SAP-Lösung, die den Anforderungen gerecht wird.

\section{Fazit}

Keine Ahnung ob das auch noch rein muss. Wenn ja fülle ich das morgen früh.

%%% SEITE 4
%% BIBLIOGRAPHY!

\newpage
\bibliographystyle{acm}

\makeliteratur{handout}
\end{document}