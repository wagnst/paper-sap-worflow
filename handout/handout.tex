\documentclass{handout}
\SetHandoutTitle{Workflowmanagement - \textbf{SAP} ERP \& \textbf{SAP} ByD}
\SetUniversitaet{Duale Hochschule Baden-Württemberg}
\SetDatum{\today}
\SetFakultaet{Angewandte Informatik IBC}
\SetSemester{Sommersemester 2014}
\SetDozent{Kai-Frank Strugalla}
\SetReferent{S. Wagner, M. Dörfer, J. Dann}
\SetMatrikelnr{8974337, 6541564, 3346893}
\SetSeminar{Proseminar Workflow (T2INF4122)}

\hypersetup{
    pdftitle={Handout \HandoutTitle},
    pdfauthor={\Referent},
    pdfsubject={},
    pdfkeywords={},
    pdfcreator={LaTeX, hyperref, KOMA-Script}, % Ersteller
  }
\begin{document}


%%% SEITE 1
\maketitle
\begin{longtable}[ht]{p{0.20\textwidth} p{0.74\textwidth}}
\multicolumn{2}{p{1\textwidth}}{Meine \underline{persönliche Definition} des Begriffes Cultural turn ist, 
\begin{quote}
	\enquote{\textit{Die (Ver-)Änderunge vom WAS und/oder WIE innerhalb der Kulturwissenschaftlichen Betrachtungen.}}
\end{quote}}
\\
\textbf{Definition (h.M.)} & Abkehr von einer Kultur der Elite hinzu einer Populärkultur.\\
\textbf{Beginn} & \textit{strittig} irgendwo zwischen dem frühen 20. Jahrhundert und 1940/50\\
\textbf{Höhepunkt} & bisher ca. 1960 mit Gründung der Kulturwissenschaften \\
\textbf{Ende} & meiner Ansicht nach gibt es keine Ende, da der Begriff des Cultural turn ein Oberbegriff für alle turns ist und sich ständig neue turns bilden b.z.w. entwickeln.\\
\textbf{Ursprung} & \textit{strittig} vom Linguistic turn oder von den sich geänderten Fragestellungen innerhalb der Sozialwissenschaften hergeleitet.\\
\textbf{Turns} & 
\begin{noindlist}
	\item Linguistic turn 
	\item Postcolonial turn \cite{SAPAbout}
	\item Iconic turn
	\item Performative turn
	\item Spatial turn
	\item Emotional turn
	\item Reflexive turn
	\item Translational turn
	\item Interpretive turn
	\item (Gender turn)
\end{noindlist}\\
\end{longtable}

\newpage

%%% SEITE 2
\begin{longtable}[ht]{p{0.20\textwidth} p{0.74\textwidth}}
\multicolumn{2}{p{1\textwidth}}{Meine \underline{persönliche Definition} des Begriffes Cultural turn ist, 
\begin{quote}
	\enquote{\textit{Die (Ver-)Änderunge vom WAS und/oder WIE innerhalb der Kulturwissenschaftlichen Betrachtungen.}}
\end{quote}}
\\
\textbf{Definition (h.M.)} & Abkehr von einer Kultur der Elite hinzu einer Populärkultur.\\
\textbf{Beginn} & \textit{strittig} irgendwo zwischen dem frühen 20. Jahrhundert und 1940/50\\
\textbf{Höhepunkt} & bisher ca. 1960 mit Gründung der Kulturwissenschaften \\
\textbf{Ende} & meiner Ansicht nach gibt es keine Ende, da der Begriff des Cultural turn ein Oberbegriff für alle turns ist und sich ständig neue turns bilden b.z.w. entwickeln.\\
\textbf{Ursprung} & \textit{strittig} vom Linguistic turn oder von den sich geänderten Fragestellungen innerhalb der Sozialwissenschaften hergeleitet.\\
\textbf{Turns} & 
\begin{noindlist}
	\item Linguistic turn
	\item Postcolonial turn
	\item Iconic turn
	\item Performative turn
	\item Spatial turn
	\item Emotional turn
	\item Reflexive turn
	\item Translational turn
	\item Interpretive turn
	\item (Gender turn)
\end{noindlist}\\
\end{longtable}

\newpage

%%% SEITE 3
\subsection{Schnittstellen}

Die mit dem Workflowbuilder gebauten Geschäftsprozesse bieten verschiedene Schnittstellen um mit SAP eigenen Systemen oder externen Tools zu kommunizieren:

\begin{itemize}
\item Kommunikation zwischen verschiedenen SAP-Systemen durch ABAP-Coding
\item Export in internes XML-Format
\item Export in BPML
\end{itemize}

BPML wird auch von anderen Workflow-Systemen, wie zum Beispiel jBPMN, Camuda BMP oder ARIS, verwendet.

\section{ByDesign}

SAP Business ByDesign (ByD) ist eine onDemand Cloudlösung für SME, also für kleine bis mittelständige Unternehmen. OnDemand bedeutet hier, dass ByD an den Bedarf der Firma angepasst wird. So bezahlt man monatlich je nach Anzahl der Nutzerlizenzen, die gerade im Gebrauch sind, mehr oder weniger. Das System wird bei SAP gehostet und die Installation, Wartung und Aktualisierung ist durch das integrierte Betriebsmodell durch SAP gewährleistet. Aufgerufen wird ByD über den Browser. Die Website ist auch außerhalb des Firmennetzes erreichbar, sodass User von überall arbeiten können.

\subsection{Grenzen von ByDesign}

Da ByD eine standardisierte Cloudlösung ist tun sich mehrere Probleme auf:

\begin{itemize}
\item Nur zu einem bestimmten Grad granulares Customizing möglich
\item Funktionalität in Module unterteilt (teils für den eigenen Betrieb unnötige Funktionalität)
\item Keine Erweiterbarkeit durch eigene Geschäftsprozesse oder SAP Customizing
\end{itemize}

Wenn die Kundenanforderungen die Grenzen von ByD überschreiten gibt es sicher eine andere SAP-Lösung, die den Anforderungen gerecht wird.

\section{Fazit}

Keine Ahnung ob das auch noch rein muss. Wenn ja fülle ich das morgen früh.

%%% SEITE 4

\newpage
\bibliographystyle{acm}

\makeliteratur{handout}
\end{document}