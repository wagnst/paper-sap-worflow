\documentclass{handout}
\SetHandoutTitle{Workflowmanagement - \textbf{SAP} ERP \& \textbf{SAP} ByD}
\SetUniversitaet{Duale Hochschule Baden-Württemberg}
\SetDatum{\today}
\SetFakultaet{Angewandte Informatik IBC}
\SetSemester{Sommersemester 2014}
\SetDozent{Kai-Frank Strugalla}
\SetReferent{S. Wagner, M. Dörfer, J. Dann}
\SetMatrikelnr{8974337, 6541564, 3346893}
\SetSeminar{Proseminar Workflow (T2INF4122)}

\hypersetup{
    pdftitle={Handout \HandoutTitle},
    pdfauthor={\Referent},
    pdfsubject={},
    pdfkeywords={},
    pdfcreator={LaTeX, hyperref, KOMA-Script}, % Ersteller
  }
\begin{document}

%%% TEIL 1
%%% STEFFEN
\maketitle

\section{SAP AG}
Die \emph{SAP AG} wurde 1972 von 5 ehemaligen IBM Mitarbeitern gegründet und ist als weltweit viertgrößter Softwarehersteller \cite{SAPFacts} der Marktführer im Bereich betriebswirtschaftlicher Standardsoftware.

\small
\begin{tabular}{L}
66.500+ Mitarbeiter weltweit \cite{SAPAtGlance} \\
253.500+ Kunden in 188 Ländern \cite{SAPAtGlance} \\
16,82 Milliarden Euro Umsatz in Q4/2013 \cite{SAPFacts} \\
\end{tabular}
\normalsize

SAP hat vier Haupt-Softwareprodukte im Portfolio. Die Produkte decken die Bereiche von Enterprise Resource Planning (ERP), Supply Chain Management (SCM), Product Lifecycle Management (PLM), Supplier Relationship Management (SRM) und Customer Relationship Management (CRM) auf verschiedene Arten ab.

\small
\begin{tabular}{L ll}
SAP \emph{Business One} & $\rightarrow$ & OnPremise / OnDemand / OnDevice \\
SAP \emph{Business ByDesign} & $\rightarrow$ & OnDemand \\
SAP \emph{Business All-in-One} & $\rightarrow$ & OnPremise / OnDemand / OnDevice\\
SAP \emph{Business Suite} & $\rightarrow$ & OnPremise \\
\end{tabular}
\normalsize

\section{SAP Basis}
Um SAP Systeme betreiben zu können, muss eine Grundlage an Technik (Hardware, Betriebssystem, Datenbank) vorhanden sein. Wenn diese Technik eine SAP-Applikation betreibt, so spricht man von SAP Basis \cite{SAPTEC}. Diese besteht aus...

\small
\begin{tabular}{L l}
...Präsentationsschicht & $\rightarrow$ SAP GUI (Windows/Web/Java) für den Anwender\\
...Applikationsschicht & $\rightarrow$ SAP NetWeaver und entsprechende Programmmodule\\
...Datenbankschicht & $\rightarrow$ Datenbank(en), wie z.B. SAP \emph{HANA} oder Oracle\\
\end{tabular}
\normalsize

%%% TEIL 2
%%% MARCO
\section{SAP Workflow Builder}
Der SAP Workflow Builder ist eine Transaktion (\texttt{SWDD}) innerhalb des SAP Systems zum grafischen Abbilden und Modellieren von Geschäftsprozessen. Dies ermöglicht den Kunden der SAP AG, ihre Produkte selbst an die eigene, individuellen Prozessabläufe anzupassen. Im Gegensatz zu anderen Lösungen zum Abbilden von Prozessen hat diese durch ihre direkte Einbindung ins SAP System einige wichtige Zusatzfunktionen, welche ihr ermöglichen, den Grad der Automatisierung zu maximieren \cite{SAPHelpWf}:

\small
\begin{tabular}{L}
Problemloser Datenzugriff\\
Warten auf interne Ereignisse\\
Zusammenarbeit mit anderen Transaktionen möglich\\
Steuerung interner Programmabläufe\\
Starten von Prozessen externer Programme
\end{tabular}
\normalsize

\subsection{Funktionsweise}
Um einen Geschäftsprozess im eigenen System umzusetzen, muss der Benutzer ihn lediglich per \textit{Drag \& Drop} grafisch modellieren. hierfür stellt der SAP Workflow Builder folgende Schritttypen bereit:

\small
	\begin{table}[h]
	\begin{center}
		\begin{tabular}{c|l}
				Symbol & \multicolumn{1}{l}{Schrittyp}\\
				\hline 
				\includegraphicstotab[width=0.4cm]{../grafiken/aktivitaet.png} & Aktivität \\
				\hdashline
				\includegraphicstotab[width=0.4cm]{../grafiken/web-aktivitaet.png} & Web-Aktivität\\
				\hdashline
				\includegraphicstotab[width=0.4cm]{../grafiken/mail-versenden.png} & Mail-Versendung\\
				\hdashline
				\includegraphicstotab[width=0.4cm]{../grafiken/formular.png} & Formularschritt\\
				\hdashline
				\includegraphicstotab[width=0.4cm]{../grafiken/benutzerentscheidung.png} & Benutzerentscheidung\\
				\hdashline
				\includegraphicstotab[width=0.4cm]{../grafiken/dokument-aus-vorlage.png} & Dokument aus Vorlage\\
				\hdashline
				\includegraphicstotab[width=0.4cm]{../grafiken/bedingung.png} & Bedingung\\
				\hdashline
				\includegraphicstotab[width=0.4cm]{../grafiken/mehrfachbedingung.png} & Mehrfachbedingung\\
				\hdashline
				\includegraphicstotab[width=0.4cm]{../grafiken/ereigniserzeuger.png} & Ereigniserzeuger\\
				\hdashline
		\end{tabular}
		\qquad
		\begin{tabular}{c|l}
				Symbol & \multicolumn{1}{l}{Schrittyp}\\
				\hline 
				\includegraphicstotab[width=0.4cm]{../grafiken/warten.png} & Warteschritt\\
				\hdashline
				\includegraphicstotab[width=0.4cm]{../grafiken/containeroperationen.png} & Containeroperationen\\
				\hdashline
				\includegraphicstotab[width=0.4cm]{../grafiken/ablaufsteuerung.png} & Ablaufsteuerung\\
				\hdashline
				\includegraphicstotab[width=0.4cm]{../grafiken/schleife.png} & Schleifen\\
				\hdashline
				\includegraphicstotab[width=0.4cm]{../grafiken/paralleler-abschnitt.png} & Paralleler Abschnitt\\
				\hdashline
				\includegraphicstotab[width=0.4cm]{../grafiken/ad-hoc-anker.png} & Ad-hoc-Anker\\
				\hdashline
				\includegraphicstotab[width=0.4cm]{../grafiken/block.png} & Block\\
				\hdashline
				\includegraphicstotab[width=0.4cm]{../grafiken/lokaler-workflow.png} & Lokaler Workflow\\
				\hdashline
		\end{tabular}		
		\end{center}
	\end{table}
\normalsize

Die meisten der Schritte sind durch ihre Namen selbsterklärend, dennoch sollen einige hier genauer erklärt werden:

Der am häufigsten verwendete Schritt ist eine sogenannte \textbf{Aktivität}. Diese kann unter anderen bestimmte Programme ausführen oder Daten verarbeiten, indem ihr in der Konfiguration eine, meist schon im System vorhandene Aufgabe, wie \textit{Material anlegen}, zugewiesen wird. Ist diese noch nicht existent, so muss sie angelegt werden.

%%% TEIL 3
%%% JONAS
\subsection{Schnittstellen}

Die mit dem Workflowbuilder gebauten Geschäftsprozesse bieten verschiedene Schnittstellen um mit SAP eigenen Systemen oder externen Tools zu kommunizieren:

\small
\begin{tabular}{L l}
Kommunikation zwischen verschiedenen SAP-Systemen durch ABAP-Coding\\
Export in internes XML-Format\\
Export in BPML\\
\end{tabular}
\normalsize

BPML wird auch von anderen Workflow-Systemen, wie zum Beispiel jBPMN, Camuda BMP oder ARIS, verwendet.

\section{SAP Business ByDesign}

SAP Business ByDesign (ByD) ist eine onDemand Cloudlösung für SME, also für kleine bis mittelständige Unternehmen. OnDemand bedeutet hier, dass ByD an den Bedarf der Firma angepasst wird. So bezahlt man monatlich je nach Anzahl der Nutzerlizenzen, die gerade im Gebrauch sind, mehr oder weniger. Das System wird bei SAP gehostet und die Installation, Wartung und Aktualisierung ist durch das integrierte Betriebsmodell durch SAP gewährleistet. Aufgerufen wird ByD über den Browser. Die Website ist auch außerhalb des Firmennetzes erreichbar, sodass Endanwender von überall arbeiten können.

\subsection{Grenzen von ByDesign}

Da ByD eine standardisierte Cloudlösung ist tun sich mehrere Probleme auf:

\small
\begin{tabular}{L l}
Nur zu einem bestimmten Grad granulares Customizing möglich\\
Funktionalität in Module unterteilt (teils für den eigenen Betrieb unnötige Funktionalität)\\
Keine Erweiterbarkeit durch eigene Geschäftsprozesse oder SAP Customizing\\
\end{tabular}
\normalsize

Wenn die Kundenanforderungen die Grenzen von ByD überschreiten gibt es sicher eine andere SAP-Lösung, die den Anforderungen gerecht wird.

%%% SEITE 4
%% BIBLIOGRAPHY!

\newpage
\bibliographystyle{acm}

\makeliteratur{handout}
\end{document}