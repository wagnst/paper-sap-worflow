\section{Large Enterprises}
%%%%%%%%%%%%%%%%%%%%%%%%%%%%
%% KAPITEL Business SUITE %%
%%%%%%%%%%%%%%%%%%%%%%%%%%%%
%% STEFFEN
%%%%%%%%%%%%%%%%%%%%%%%%%%%%
\subsection{SAP R/3 Business Suite}
\label{sec:business-suite}

\section{Small and Medium Enterprises}
%%%%%%%%%%%%%%%%%%%%%%%%%%
%% KAPITEL Business One %%
%%%%%%%%%%%%%%%%%%%%%%%%%%
%% MARCO
%%%%%%%%%%%%%%%%%%%%%%%%%%
\subsection{SAP Business One}
\label{sec:business-one}

%%%%%%%%%%%%%%%%%
%% KAPITEL ByD %%
%%%%%%%%%%%%%%%%%
%% JONAS
%%%%%%%%%%%%%%%%%
\subsection{SAP Business By Design}
\label{sec:byd}

SAP Business ByDesign ist eine ERP OnDemand Cloudlösung für kleine bis mittelständige Unternehmen ab 25 Mitarbeitern. Sie ist preiswert und skalierbar, da auf monatlicher Basis bezahlt wird und Nutzerlizenzen dynamisch hinzugekauft werden können. Weiterhin wird die Software sehr schnell bereitgestellt und der Kunde hat keine weiteren IT-Aufwendungen, da das System bei SAP in einem Rechenzentrum gehostet wird. ByDesign enthält dabei alle nötigen vorkonfigurierten Workflowprozesse, von Verwaltung der Kundenbeziehungen, Beschaffung und Lieferketten, bis hin zu Rechnungswesen und Werbeplanung. Trotzdem verliert der Kunde keine Flexibilität gegenüber den standardmäßigen SAP-ERP Lösungen, da der Lösungsumfang sehr genau konfiguriert werden kann. Somit können unnötige Funktionalitäten abgeschaltet werden, um dem Endnutzer die Arbeit mit dem System so einfach wie möglich zu gestalten.

%%%%%%%%%%%%%%%%%%%%%%
%% KAPITEL ALLINONE %%
%%%%%%%%%%%%%%%%%%%%%%
%% MARCO
%%%%%%%%%%%%%%%%%%%%%%
\subsection{SAP All-in-One}
\label{sec:allinone}

%%%%%%%%%%%%%%%%%%%%%%%
%% KAPITEL VERGLEICH %%
%%%%%%%%%%%%%%%%%%%%%%%
%% STEFFEN
%%%%%%%%%%%%%%%%%%%%%%%
\subsection{Vergleich der Produkte}
\begin{table}[h]
\begin{center}
\begin{tabular}{p{3.5cm}||p{3cm}|p{3cm}|p{3cm}}
  \emph{\gls{sap} \gls{sme} Lösung} & \emph{\gls{sap} Business One (\ref{sec:business-one})} & \emph{\gls{sap} \gls{byd} (\ref{sec:byd})} & \emph{\gls{sap} All-In-One (\ref{sec:allinone})}\\	
  \hline
  kurze Beschreibung & Eine einzelne, integrierte Anwendung mit der man ein gesamtes Unternehmen verwalten kann & Die Beste On-Demand Lösung von SAP & Umfassende, integrierte und sehr einfach als \gls{saas} konfiguriert\\
  \hline
  Anzahl der Nutzer & bis zu 100 & 100 bis 500 &  bis zu 2.500\\
  \hline
  Länderverfügbarkeit & 40 Länder & US, UK, D, F, Indien, China & 50 Länder\\
  \hline	
  Implementierungsart & On-Premises & On-Demand & On-Premises oder Hosted\\
  \hline	
  Implementierungszeit & 2-8 Wochen & 4-8 Wochen & 8-16 Wochen\\
  \hline	
  Transaktionsvolumen & niedrig & mittel & hoch\\
  \hline	
  Industrielösungen & mehrere & wenige & viele\\
  \hline				
\end{tabular}
\end{center}
% Beschriftung festlegen:
\caption{Vergleich der \gls{sap} \gls{sme} Produkte} 
% ein Label definieren, mit dessen Hilfe man (an beliebiger Stelle im Dokument) Bezug nehmen kann:
\label{tab:smevergleich}
\end{table}

Tabelle \ref{tab:smevergleich} zeigt ein Vergleich zwischen den verschiedenen Produkten, \gls{sap} Business One (\ref{sec:business-one}), \gls{sap} \gls{byd} (\ref{sec:byd}) und zum Schluss noch \gls{sap} All-In-One (\ref{sec:allinone}). Neben einer kurzen Beschreibung zu dem Produkt, finden sich in dieser Tabelle auch die geeigneten Nutzer- bzw. Mitarbeiterzahlen, die Länderverfügbarkeit und andere Vergleiche wie die Implementierungszeit. Hier erkennt man auch wieder wie verschieden die Produkte doch sind, was viele potentielle Kunden nicht unbedingt gleich vermuten. So ist die \gls{sap} \gls{byd}-Lösung zum Beispiel nur in sechs Ländern verfügbar, wohingegen die anderen beiden in 40 und in 50 Ländern verfügbar sind \cite{SAPin24hrs}.