\section{Large Enterprises}
%%%%%%%%%%%%%%%%%%%%%%%%%%%%
%% KAPITEL Business SUITE %%
%%%%%%%%%%%%%%%%%%%%%%%%%%%%
%% STEFFEN
%%%%%%%%%%%%%%%%%%%%%%%%%%%%
\subsection{SAP R/3 Business Suite}
\label{sec:business-suite}

\section{Small and Medium Enterprises}
%%%%%%%%%%%%%%%%%%%%%%
%% KAPITEL ALLINONE %%
%%%%%%%%%%%%%%%%%%%%%%
%% JONAS
%%%%%%%%%%%%%%%%%%%%%%
\subsection{SAP All-in-One}
\label{sec:allinone}

Die \gls{sap} All-in-One Lösung bietet ein \gls{sap} \gls{erp} und \gls{sap} \gls{nw} für mittelständische Unternehmen. Ein Basissystem ist schon ab 90.000 Euro erhältlich und lässt sich nach den Wünschen der Kunden skalieren.

All-in-One basiert auf vordefinierten, branchenspezifischen Geschäftsprozessen. Diese wurden mit der langjährige Erfahrung der \gls{sap} im Bereich Unternehmenssoftware entwickelt. Dadurch lassen sich All-in-One Systeme schnell aufsetzen und erzeugen keine unnötigen Kosten. Der Kunde muss trotzdem nicht auf Flexibilität verzichten, da die Geschäftsprozesse genau an die Bedürfnisse der Firma angepasst werden können.

\gls{sap} All-in-One kann durch spezifische Lösungen erweitert und noch spezieller auf das eigene Unternehmen zugeschnitten werden.

Branchenlösungen sind vorhanden für Automobilzulieferer, Komponentenfertiger, Kleinserienfertiger, Kunststoffverarbeiter und Metallverarbeiter \cite{AiOBeratung}.

All-in-One ist gedacht um die Kernprozesse des Unternehmens zu automatisieren und so die Innovations- und Wachstutmsfähigkeit des Unternehmens zu erhöhen.

%%%%%%%%%%%%%%%%%
%% KAPITEL ByD %%
%%%%%%%%%%%%%%%%%
%% JONAS
%%%%%%%%%%%%%%%%%
\subsection{SAP Business By Design}
\label{sec:byd}

\gls{sap} \gls{byd} ist eine \gls{erp} \gls{ondemand} Cloudlösung für \gls{sme} ab 25 Mitarbeitern. Die Nutzung ist preiswert und skalierbar, da auf monatlicher Basis bezahlt wird und Nutzerlizenzen dynamisch hinzugekauft werden können. Die Software wird schnell bereitgestellt und der Kunde hat keine weiteren IT-Aufwendungen, da das System bei \gls{sap} direkt im Rechenzentrum gehostet wird.

\gls{byd} enthält dabei alle nötigen vorkonfigurierten Workflowprozesse, von Verwaltung der Kundenbeziehungen, Materialbeschaffung und Lieferkettenverwaltung, bis hin zu Rechnungswesen und Werbeplanung. Trotzdem verliert der Kunde kaum Flexibilität gegenüber den etablierten \gls{sap}-\gls{erp} Lösungen, wie \gls{zb} \gls{sap} Business One (siehe \ref{sec:business-one}), da der Lösungsumfang konfiguriert werden kann, um ein möglichst breites Spektrum an Aufgaben abdecken zu können. Jedoch bietet \gls{byd} kein eigentliches Customizing \cite{ERP4Students}, da die einzelnen Geschäftsprozesse nur noch geringfügig den Bedürfnissen der Firma angepasst werden können.

%%%%%%%%%%%%%%%%%%%%%%%%%%
%% KAPITEL Business One %%
%%%%%%%%%%%%%%%%%%%%%%%%%%
%% JONAS
%%%%%%%%%%%%%%%%%%%%%%%%%%
\subsection{SAP Business One}
\label{sec:business-one}

Business One ist die dritte \gls{sap}-Lösungen für \gls{sme}. Sie wird im \gls{ondemand}- oder Vor-Ort-Modell unterstützt. Stellt also eine Art Mittelweg zwischen All-in-One(\ref{sec:allinone}) und \gls{byd} dar. Wenn ein schneller Datenzugriff bereitgestellt werden muss läuft \gls{sap} Business One auch auf der In-Memory-Computing-Plattform \gls{sap} HANA.

\gls{sap} und seine Partner stellen für Business One über 550 Branchenlösungen mit vorkonfigurierten Workflows bereit. Somit kauft der Kunde eine Lösung, die schon von vielen Unternehmen genutzt wird. Dadurch werden natürlich Kosten und Risiken gesenkt, da mögliche Probleme bereits vorher aufgetreten sind und somit schnell und kostengünstig gelöst werden können.

Natürlich sind auch hier alle Workflows konfigurierbar und können über unternehmensspezifisches Customizing in nur 2 - 8 Wochen auf den Kunden zugeschnitten werden.\cite{BusinessOne}

In Business One können alle Prozesse eines Unternehmens abgebildet werden und die Mitarbeiter haben sogar externen Zugriff auf das System via \gls{sap} mobile Apps.

%%%%%%%%%%%%%%%%%%%%%%%
%% KAPITEL VERGLEICH %%
%%%%%%%%%%%%%%%%%%%%%%%
%% STEFFEN
%%%%%%%%%%%%%%%%%%%%%%%
\subsection{Vergleich der Produkte}
\begin{table}[H]
\begin{center}
\begin{tabular}{p{3.8cm}||p{3cm}|p{3cm}|p{3cm}}
  \emph{\gls{sap} \gls{sme} Lösung} & \emph{\gls{sap} Business One (\ref{sec:business-one})} & \emph{\gls{sap} \gls{byd} (\ref{sec:byd})} & \emph{\gls{sap} All-In-One (\ref{sec:allinone})}\\	
  \hline
  kurze Beschreibung & Eine einzelne, integrierte Anwendung mit der man ein gesamtes Unternehmen verwalten kann & Die Beste \gls{ondemand} Lösung von SAP & Umfassende, integrierte und sehr einfach als \gls{saas} konfiguriert\\
  \hline
  Anzahl der Nutzer & bis zu 100 & 100 bis 500 &  bis zu 2.500\\
  \hline
  Länderverfügbarkeit & 40 Länder & US, UK, D, F, Indien, China & 50 Länder\\
  \hline	
  Implementierungsart & \gls{onpremise} & \gls{ondemand} & \gls{onpremise} oder Hosted\\
  \hline	
  Implementierungszeit & 2-8 Wochen & 4-8 Wochen & 8-16 Wochen\\
  \hline	
  Transaktionsvolumen & niedrig & mittel & hoch\\
  \hline	
  Industrielösungen & mehrere & wenige & viele\\
  \hline				
\end{tabular}
\end{center}
% Beschriftung festlegen:
\caption{Vergleich der \gls{sap} \gls{sme} Produkte} 
% ein Label definieren, mit dessen Hilfe man (an beliebiger Stelle im Dokument) Bezug nehmen kann:
\label{tab:smevergleich}
\end{table}

Tabelle \ref{tab:smevergleich} zeigt ein Vergleich zwischen den verschiedenen Produkten, \gls{sap} Business One (\ref{sec:business-one}), \gls{sap} \gls{byd} (\ref{sec:byd}) und zum Schluss noch \gls{sap} All-In-One (\ref{sec:allinone}). Neben einer kurzen Beschreibung zu dem Produkt, finden sich in dieser Tabelle auch die geeigneten Nutzer- bzw. Mitarbeiterzahlen, die Länderverfügbarkeit und andere Vergleiche wie die Implementierungszeit. Hier erkennt man auch wieder wie verschieden die Produkte doch sind, was viele potentielle Kunden nicht unbedingt gleich vermuten. So ist die \gls{sap} \gls{byd}-Lösung zum Beispiel nur in sechs Ländern verfügbar, wohingegen die anderen beiden in 40 und in 50 Ländern verfügbar sind \cite{SAPin24hrs}.
