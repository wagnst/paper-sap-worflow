%%%%%%%%%%%%%%%%%%%
%% KAPITEL Intro %%
%%%%%%%%%%%%%%%%%%%
\section{Einführung}
% JONAS
% was ist es, wer benutzt es, wofür braucht man es

\gls{sap} \gls{byd} ist eine \gls{erp} \gls{ondemand} Cloudlösung. Die Nutzung wird monatlich bezahlt. Dadurch können Nutzerlizenzen dynamisch erworben werden und der Kunde bezahlt immer nur so viel, wie er muss.

\gls{byd} ist preiswert und skalierbar. Die Software wird innerhalb weniger Wochen bereitgestellt. Außerdem wird das System direkt bei \gls{sap} vor Ort im Rechenzentrum gehostet, sodass der Kunde keine weiteren IT-Investitionen tätigen muss.

\gls{byd} enthält alle nötigen vorkonfigurierten Geschäftsprozesse, von Verwaltung der Kundenbeziehungen, Materialbeschaffung und Lieferkettenverwaltung, bis hin zu Rechnungswesen und Werbeplanung. Trotzdem verliert der Kunde kaum Flexibilität gegenüber den etablierten \gls{sap}-Lösungen, wie \gls{zb} \gls{sap} Business One (siehe \ref{sec:business-one}).

Für Installation, Wartung und Aktualisierung der Lösung sorgt das integrierte Betriebsmodell. Alle Betriebskosten, die durch ein Vor-Ort System entstehen sind also im Preis einbegriffen. Damit kann sich der Kunde vollständig auf sein Kerngeschäft konzentrieren.

\gls{sap} \gls{byd} wird über eine sichere Internetverbindung und einen Webbrowser als dynamische Website aufgerufen. Somit können Mitarbeiter von überall auf ihren Arbeitsplatz zugreifen und müssen weder vor Ort im Büro sein oder sich anderweitig ins Firmennetz einwählen.

\subsubsection{Vorteile von ByD}

\begin{itemize}
\item Business ByDesign vereinigt alle Vorteile einer modernen Unternehmensanwendung, bei minimalen Anforderungen an die IT
\item SAP Business ByDesign greift auf bewährte Geschäftsvorfälle zu, die umgehend einsatzbereit sind
\item Der Kunde nutzt automatisch stets die aktuellste Softwareversion
\item SAP Business ByDesign schont die Investition für eine eigene IT-Infrastruktur, durch ein skalierbares Mietmodell
\item Wechselnde Geschäftsanforderungen gehen mit der Nutzung der Softwarebereiche Hand in Hand
\end{itemize}
\cite{itelligence}

%%%%%%%%%%%%%%%%%%%%%
%% KAPITEL HandsOn %%
%%%%%%%%%%%%%%%%%%%%%
%% JONAS
%%%%%%%%%%%%%%%%%%%%%
\section{Hands On}

\subsection{Beispielworkflow}
\label{sec:byd-bsp}

\subsubsection{Vorstellung des Workflows}
\label{sec:byd-bsp-vorstellung}
% Schulungsworkflow beschreiben (anwendersicht)

\subsubsection{Umsetzung des Workflows}
\label{sec:byd-bsp-umsetzung}
% technische sicht, "`klickbares"' howto

%%%%%%%%%%%%%%%%%%%%%
%% KAPITEL Grenzen %%
%%%%%%%%%%%%%%%%%%%%%
%% JONAS 
%%%%%%%%%%%%%%%%%%%%%
\section{Grenzen von ByD}

Trotz der ganzen Vorteile von \gls{byd} stößt auch diese Lösung, wie alle anderen, an ihre Grenzen.

\subsubsection{Vordefinierte Geschäftsprozesse}

Durch die Idee hinter \gls{byd}, eine vorkonfigurierte On-Demand Unternehmensmanagement Applikation bereitzustellen, weißt es Nachteile gegenüber den anderen \gls{sme}-Lösungen im Bereich Customizing auf. So kann \gls{byd} nicht beliebig eingestellt werden.

\subsubsection{Module}

Da \gls{byd} in Form von Modulen zusammengestellt wird bekommt der Kunde unausweichlich auch Funktionalität, die er gar nicht benötigt und bezahlt für unnötige Anwendungsbestandteile. In diesem Aspekt sind Business One \ref{sec:business-one} oder \gls{sap} All-in-One \ref{sec:allinone} die bessere Wahl.

\subsubsection{Erweiterbarkeit}

Im Gegensatz zu den beiden anderen \gls{sme}-Systemen kann \gls{byd} nicht beliebig erweitert werden. So können nicht einfach spezifische Prozesse neu entwickelt und in das vorhandene System eingebunden werden, da \gls{byd} keine Möglichkeit bietet eigene Workflows anzulegen und auch \gls{sap} keine weiteren Add-Ons anbietet, als die Standardsoftware.

