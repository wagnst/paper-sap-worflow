%%%%%%%%%%%%%%%%%%%%
%% KAPITEL SAP AG %%
%%%%%%%%%%%%%%%%%%%%
Die, 1972 von fünf ehemaligen \gls{ibm}-Mitarbeitern gegründete, \gls{sap} AG ist als weltweit viertgrößter Softwarehersteller (Stand Q4/2013, \cite{SAPFacts}) der Marktführer im Bereich betriebswirtschaftlicher Standardsoftware. Mit weltweit mehr als 66.500
Mitarbeitern (Stand Q4/2013, \cite{SAPAtGlance}) und über 253.500 Kunden in 188 Ländern (Stand Q4/2013, \cite{SAPAtGlance}) erwirtschaftet sie einen jährlichen Umsatz von ca. 16,82 Milliarden \euro (Euro) (Stand Q4/2013, \cite{SAPFacts}). Tabelle \ref{tab:SAPKennzahlen} zeigt die Entwicklung wichtiger Kennzahlen der SAP AG \cite{SpringerControllingSAP}.

%% Tabelle mit SAP Kennzahlen, aktualisiert Juni 2014
% der optionale Parameter "h" gibt an, dass der Block
% mit der Abbildung vorzugsweise an der aktuellen Position,
% alternativ unten ("botton") platziert werden soll
\begin{table}[H]
\begin{center}
\begin{tabular}{l||l|l|l|l|l|l}
  & \emph{2002} & \emph{2004} & \emph{2006} & \emph{2008} & \emph{2010} & \emph{2013}\\	
  \hline
  Umsatz (in Mio. \euro) & 7.413 & 7.514 & 9.402 & 11.575 & 12.464 & 16.820\\
  \hline
  Betriebsergebnis (in Mio. \euro) & 1.626 & 2.018 & 2.563 & 2.701 & 2.591 & 5.900\\
  \hline
  Mitarbeiter & 28.797 & 32.205 & 39.355 & 51.544 & 53.513 & 66.500\\
  \hline	
\end{tabular}
\end{center}
% Beschriftung festlegen:
\caption{Entwicklung wichtiger Kennzahlen der \gls{sap} AG} 
% ein Label definieren, mit dessen Hilfe man (an beliebiger Stelle im Dokument) Bezug nehmen kann:
\label{tab:SAPKennzahlen}
\end{table}

\gls{sap} erzielt Umsätze nicht nur mit Software. Der Anteil von Software an den Gesamtumsätzen macht lediglich 26\% aus. Daneben spielen insbesondere die Bereiche Support und Beratung eine große Rolle. Abbildung \ref{abb:SAPUmsatzverteilung} zeigt die Verteilung der Umsätze im Jahr 2010 auf einzelne Bereiche der \gls{sap} AG.

%%Kuchendiagramm Verteilung der Umsätze auf einzelne Bereiche der SAP AG 2010
\begin{figure}[H]
  \centering 
  \begin{tikzpicture} 
    \pie[text=legend,radius=2]{49/Support, 26/Software, 18/Beratung, 7/Sonstige} 
  \end{tikzpicture} 
  \caption{Verteilung der Umsätze auf einzelne Bereiche der \gls{sap} AG} 
  \label{abb:SAPUmsatzverteilung} 
\end{figure} 

Neben dem Firmenhauptsitz Walldorf existieren noch Niederlassungen in über 130 Ländern \cite{SAPLocations} rund um den Globus.
Das Produktportfolio der SAP AG enthält Lösungen für alle zentralen Geschäftsabläufe in Firmen. Dazu gehören unter anderem \gls{erp} (siehe \ref{sec:erp-definition}), \gls{crm} (siehe \ref{sec:crm-definition}), \gls{srm} (siehe \ref{sec:srm-definition}), \gls{scm} (siehe \ref{sec:scm-definition}) oder \gls{plm} (siehe \ref{sec:plm-definition}) Systeme.

