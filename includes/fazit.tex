% irgendwie einmal ByD und Workflow Builder zusammenfassen (jeweils Jonas & Marco)
% ein kleinen Abschnitt dass ByD f�r Firmen besser ist, die mit fest definierten Prozessen einverstanden sind
% und keine eigenen brauchen, als auch f�r kleine Firmen. F�r gro�e, die viel Customizing brauchen
% ist Workflow Builder weitaus besser, da viel mehr Individualisierbar und Komplexer
% jeder Wunsch kann damit erf�llt werden. Perfekte Implementierung in vorhandene SAP Landschaft

Vor allem für kleine Firmen, die kein erweitertes Customizing benötigen, da sie Standardprozesse verwenden, ist der \gls{sap} Workflowbuilder ungeeignet. Zu hoch sind die Kosten der Entwicklung eigener Geschäftsprozesse und vor allem der Aufwand eine komplette \gls{sap} Landschaft ins Firmenumfeld zu integrieren. Eine komplexe \gls{sap} Business Suite (Kapitel \ref{sec:business-suite}) Installation kann nämlich bis zur endgültigen Inbetriebnahme dauern. Dies ist für kleine Firmen vor allem in Bezug auf Kosten nicht vertretbar. Handelt es sich jedoch um Großfirmen, welche ganz eigene Geschäftsabläufe definiert haben und von diesen auch keinesfalls abweichen können (\gls{zb}, dass Pakete immer von \gls{hub} A zu \gls{hub} B transportiert werden, bevor zum Kunde ausgeliefert werden).

Hier kommt \gls{sap} \gls{byd} ins Spiel. Durch die viel schnellere und kostengünstigere Bereitstellung des Systems und die vorkonfigurierten Workflows können auch Firmen mit kleiner oder gar keiner IT-Abteilung ein \gls{sap}-System nutzen. \gls{sap} stellt die Serverarchitektur und kümmert sich um deren Verwaltung und der Kunde kann sich ganz auf sein Kerngeschäft konzentrieren. Somit können viele Kosten für IT gespart werden.

