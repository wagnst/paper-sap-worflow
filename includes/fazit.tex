% irgendwie einmal ByD und Workflow Builder zusammenfassen (jeweils Jonas & Marco)
% ein kleinen Abschnitt dass ByD f�r Firmen besser ist, die mit fest definierten Prozessen einverstanden sind
% und keine eigenen brauchen, als auch f�r kleine Firmen. F�r gro�e, die viel Customizing brauchen
% ist Workflow Builder weitaus besser, da viel mehr Individualisierbar und Komplexer
% jeder Wunsch kann damit erf�llt werden. Perfekte Implementierung in vorhandene SAP Landschaft

Für große Firmen, welche häufig komplexe Geschäftsprozesse mit sich bringen und von diesen, wenn überhaupt, nur in Verbindung mit großen Kosten, abweichen können (\gls{zb}, dass Pakete immer von \gls{hub} A zu \gls{hub} B transportiert werden, bevor sie zum Kunden ausgeliefert werden), ist der \gls{sap} Workflow Builder ein sehr nützliches und oft unentbehrliches Tool. Mit diesem können sie vorhandene Geschäftsprozesse, deren Daten häufig schon zum Großteil mit \gls{sap} Systemen verwaltet werden, tief im System verankert implementieren und somit maximal automatisieren. Nachdem die Mitarbeiter entsprechend geschult wurden, sind entsprechende Firmen sogar dazu in der Lage, ihre Systeme bei Aufnahme neuer oder Änderung vorhandener Workflows autark zu verwalten und anzupassen, was auf Dauer gesehen Kosten spart. 

Allerdings sind die Kosten und der Aufwand der Integration einer kompletten \gls{sap} Landschaft ins Firmenumfeld vor allem für kleine Firmen häufig nicht tragbar. Die komplexe Installation einer \gls{sap} Business Suite (Kapitel \ref{sec:business-suite}) kann nämlich vom Zeitpunkt des Auftrags bis hin zur endgültigen Inbetriebnahme bis zu einem Jahr dauern. Für diese Firmen ist es durchaus sinnvoll, die deutlich kostengünstigere und schnellere Bereitstellung des \gls{sap} \gls{byd} zu nutzen, da deren Arbeitsabläufe häufig den ins System implementierten gleichen oder sie sich ohne großen Aufwand an diese anpassen lassen. Weitere, nicht zu vernachlässigende Vorteile für kleine und mittelständige Firmen sind die Bereitstellung und Verwaltung der Serverarchitektur und Sicherstellung der Erreichbarkeit durch \gls{sap}, da so Kosten für eine große IT-Abteilung gespart werden können, ohne dass die Erreichbarkeit des Systems eingeschränkt wird.

Aus dem Produktportfolio der \gls{sap} AG lässt sich also ablesen, dass diese versucht, möglichst allen Kunden eine individuell geeignete Lösung anzubieten.

%Vor allem für kleine Firmen, die kein erweitertes \gls{customizing} benötigen, da sie Standardprozesse verwenden, ist der \gls{sap} Workflowbuilder ungeeignet. Zu hoch sind die Kosten der Entwicklung eigener Geschäftsprozesse und vor allem der Aufwand, eine komplette \gls{sap} Landschaft ins Firmenumfeld zu integrieren. Eine komplexe \gls{sap} Business Suite (Kapitel \ref{sec:business-suite}) Installation kann nämlich bis zur endgültigen Inbetriebnahme bis zu einem Jahr dauern. Dies ist für kleine Firmen, vor allem in Bezug auf die Kosten, nicht vertretbar. Handelt es sich jedoch um Großfirmen, welche ganz eigene Geschäftsabläufe definiert haben und von diesen auch keinesfalls abweichen können (\gls{zb}, dass Pakete immer von \gls{hub} A zu \gls{hub} B transportiert werden, bevor sie zum Kunden ausgeliefert werden), so sollte eher die Lösung \gls{sap} Business Suite in Betracht gezogen werden.

%\gls{sap} \gls{byd} kommt ins Spiel, wenn Firmen keine eigenen fest definierten Geschäftsabläufe haben. Durch die viel schnellere und kostengünstigere Bereitstellung des Systems und die vorkonfigurierten Workflows können auch Firmen mit kleiner oder gar keiner IT-Abteilung ein \gls{sap}-System nutzen. \gls{sap} stellt die Serverarchitektur und kümmert sich um deren Verwaltung und der Kunde kann sich ganz auf sein Kerngeschäft konzentrieren. Somit können viele Kosten für IT gespart werden.

