%%%%%%%%%%%%%%%%%%%
%% KAPITEL Intro %%
%%%%%%%%%%%%%%%%%%%
\section{Einführung}
% JONAS
% was ist es, wer benutzt es, wofür braucht man es

\subsection{Builder Funktionen}
\label{sec:builder-funktionen}
% MARCO
% welche Funktionalitäten hat der Builder...

\subsection{Builder Elemente}
\label{sec:builder-elemente}
% MARCO, STEFFEN
% tabelle mit elementen..
% was ist wofür gedacht

%%%%%%%%%%%%%%%%%%%%%
%% KAPITEL HandsOn %%
%%%%%%%%%%%%%%%%%%%%%
%% MARCO
%%%%%%%%%%%%%%%%%%%%%
\section{Hands On}

\subsection{Erster Beispielworkflow}
\label{sec:builder-1-bsp}
% kleiner sinnloser workflow (schleife,...)

\subsection{Zweiter Beispielworkflow}
\label{sec:builder-2-bsp}
%

\subsubsection{Vorstellung des Workflows}
\label{sec:builder-2-bsp-vorstellung}
% wofür ist der workflow gut, was soll er tun (aus anwendersicht)

\subsubsection{Umsetzung des Workflows}
\label{sec:builder-2-bsp-umsetzung}
% technische sicht, "`klickbares"' howto

%%%%%%%%%%%%%%%%%%%%%%%%%%
%% KAPITEL Fremdsysteme %%
%%%%%%%%%%%%%%%%%%%%%%%%%%
%% JONAS 
%%%%%%%%%%%%%%%%%%%%%%%%%%
\section{Schnittstellen}
%http://scn.sap.com/docs/DOC-31056
%Many SAP applications deliver workflows as content with the %SAP application. ERP, CRM, SRM are examples of SAP %applications that provide ready-to-use workflows. You can %change these workflows to reflect your company processes by %using the graphical workflow builder or build your own from %scratch.

\subsection{XML}
\label{sec:export-xml}
% was ist dieses Format
% in welche Programme kann man es importieren?

\gls{xml} ist die Abkürzung für E\textbf{x}tensible \textbf{M}arkup \textbf{L}anguage und bezeichnet eine Auszeichnungssprache. Durch diese können hierarchisch strukturierte Daten in Textform dargestellt werden. \gls{xml} besteht aus Elementen, deren Name relativ frei gewählt werden darf. Elemente haben einen Anfangs- (<elementName>) und einen Endtag (</elementName>) und zwischen diesen können wiederum weitere Elemente, Text oder andere Knoten enthalten sein. 

Das WorldWideWebConsortium, kurz W3C, hat \gls{xml} als eine Metasprache definiert, auf deren Basis anwendungsspezifische Auszeichnungssprachen entwickelt werden können. Diese werden beschrieben durch ein Schema, welches festlegt, welche Elemente verwendet werden dürfen und welches Verhalten diese aufweisen.

\subsection{BPML}
\label{sec:export-bpml}
% was ist dieses Format
% in welche Programme kann man es importieren?

Die \textbf{B}usiness \textbf{P}rocess \textbf{M}odeling \textbf{L}anguage (\gls{bpml}) ist genau wie XHTML auch eine Ableitung von \gls{xml}.

\subsection{SAP Fremdsysteme}
\label{sec:export-sap}
