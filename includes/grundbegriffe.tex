%%%%%%%%%%%%%%%%%
%% KAPITEL ERP %%
%%%%%%%%%%%%%%%%%
\section{Enterprise Resource Planning}
\label{sec:erp-definition}
Bei \gls{erp} Systemen handelt es sich um eine betriebswirtschaftliche Software, die in Betrieben oder Unternehmen eingesetzt werden kann. \gls{erp} IT-Systeme stehen für die Systemintegration der gesamten finanz- und warenwirtschaftlich orientierten Werschöpfungskette. Dabei umfasst es alle Teilprozesse von der strategischen und operationalen Planung über Herstellung, Distribution bis zur Steuerung von Auftragsabwicklung und Bestandsmanagement. Ein derartiges System verknüpft insbesondere Informationen über Finanzen, personelle Ressourcen, Produktion, Vertrieb und Einkauf. Es verbindet Kundendatenbanken, Auftragsverfolgung, Debitoren- und Kreditorenbuchaltung, Lagerverwaltung und vieles mehr \cite{ERPDefinition}.

%% Kuchendiagramm Marktanteile der Softwareunternehmen bei ERP Software
\begin{figure}[h]
  \centering 
  \begin{tikzpicture} 
    \pie[text=legend,radius=2]{46.8/Andere Anbieter, 25.4/SAP, 12.4/Sage, 6/infor, 5/Microsoft, 4.5/Oracle} 
  \end{tikzpicture} 
  \caption{Marktanteile der Softwareunternehmen bei \gls{erp} Software} 
  \label{abb:SAPMarktanteil} 
\end{figure} 

Im Gegensatz zu den Hauptwettbewerbern Oracle und Microsoft konzentriet sich \gls{sap} auf Unternehmenssoftware. Mit ihren \gls{erp}-Produkten erlangt sie weltweit einen Marktanteil von über 25\% (Siehe Abbildung \ref{abb:SAPMarktanteil}).

%%%%%%%%%%%%%%%%%
%% KAPITEL SCM %%
%%%%%%%%%%%%%%%%%
\section{Supply Chain Management}
\label{sec:scm-definition}
Der Ausdruck \gls{scm} bzw. Lieferkettenmanagement, deutsch auch Wert-schöpfungslehre, bezeichnet die Planung und das Management aller Aufgaben bei Lieferantenwahl, Beschaffung und Umwandlung sowie aller Aufgaben der Logistik. Insbesondere enthält es die Koordinierung und Zusammenarbeit der beteiligten Partner (Lieferanten, Händler, Logistikdienstleister, Kunden). \gls{scm} integriert Management innerhalb der Grenzen eines Unternehmens und über Unternehmensgrenzen hinweg. Wesentliches Paradigma hierbei ist es, dass nicht mehr Einzelunternehmen, sondern stattdessen vernetzte Lieferketten miteinander konkurrieren, wodurch eine Integration und Koordination der Mitglieder des Systems "`Lieferkette"' nötig wird. Diese Aufgabe übernimmt das \gls{scm} \cite{SCMDefinition}.

%%%%%%%%%%%%%%%%%
%% KAPITEL PLM %%
%%%%%%%%%%%%%%%%%
\section{Product Lifecycle Management}
\label{sec:plm-definition}
\gls{sap} \gls{plm} dient dem Verwalten und Steuern, also dem Orgranisieren und managen der Aufgaben, die sich aus dem kompletten Produkt "`Lebenszyklus"' ergeben. Es ist also darauf fokusiert Unternehmen bei der Organisation der Entwicklung von neuen Produkten zu Helfen. Von der Konstruktion und Produktion über den Vertrieb bis hin zur Demontage und dem Recycling \cite{PLMDefinition}.

%%%%%%%%%%%%%%%%%
%% KAPITEL SRM %%
%%%%%%%%%%%%%%%%%
\section{Supply Chain Management}
\label{sec:srm-definition}
\gls{srm} ist der Bereich des Supply Chain Managements, der sich mit der Auswahl, Steuerung und Kontrolle der Lieferanten beschäftigt und sich auf die spezifischen Anforderungen, die sich aus der Beschaffung von Gütern und Dienstleistungen ergeben, konzentriert. Das Ziel des Lieferantenmanagements ist die effizientere Gestaltung und Koordination der Beziehungen und Prozesse zwischen einer Organisation und deren Lieferanten \cite{SRMDefinition}.

%%%%%%%%%%%%%%%%%
%% KAPITEL CRM %%
%%%%%%%%%%%%%%%%%
\section{Customer Relationship Management}
\label{sec:crm-definition}
\gls{crm} steht für Customer Relationship Management. Es handelt sich um eine bereichs-übergreifende, IT-unterstützte Geschäftsstrategie, die auf den systematischen Aufbau und die Pflege dauerhafter und profitabler Kundenbeziehungen abzielt. Durch dieses System soll der Marktanteil eines Unternehmens erhöht und die Kundenzufriedenheit gesteigert werden. Und außerdem eine Segmentierung des Kundenstamms erreicht werden. Eine zentrale Erfassung der Daten bietet den Vorteil, Kosten zu reduzieren \cite{ERPDefinition}.

