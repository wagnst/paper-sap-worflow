%GLOSSAR

%glossaries mit Acronymen
\newglossaryentry{abap}
{
    name={ABAP},
    description={Ist eine Programmiersprache der \gls{sap} AG.},
    first={Advanced Business Application Programming (ABAP)},
    long={Advanced Business Application Programming}
}

\newglossaryentry{ui}
{
    name={UI},
    description={Bezeichnet die Bedienoberfläche eines Computerprogramms.},
    first={User Interface (UI)},
    long={User Interface}
}

\newglossaryentry{nw}
{
    name={NW},
    description={SAP NetWeaver ist ein Produkt der Firma SAP, die NetWeaver als Plattform für Geschäftsanwendungen bezeichnet. Grundlage für alle Anwendungen auf NetWeaver ist der SAP NetWeaver Application Server.},
    first={NetWeaver (NW)},
    long={NetWeaver}
}

\newglossaryentry{erp}{
		name=ERP,
		description={Ein Enterprise-Resource-Planning-System (ERP-System) unterstützt sämtliche in einem Unternehmen ablaufenden Geschäftsprozesse. Es enthält Module für die Bereiche Beschaffung, Produktion, Vertrieb, Anlagenwirtschaft, Personalwesen, Finanz- und Rechnungswesen usw., die über eine gemeinsame Datenbasis miteinander verbunden sind.},
		first={Enterprise Resource Planning (ERP)},
		long={Enterprise Resource Planning}
}

\newglossaryentry{crm}{
		name=CRM,
		description={CRM ist zu verstehen als ein strategischer Ansatz, der zur vollständigen Planung, Steuerung und Durchführung aller interaktiven Prozesse mit den Kunden genutzt wird. CRM umfasst das gesamte Unternehmen und den gesamten Kundenlebenszyklus und beinhaltet das Database Marketing und entsprechende CRM-Software als Steuerungsinstrument},
		first={Customer Relationship Managment (CRM)},
		long={Customer Relationship Managment}		
}

\newglossaryentry{srm}{
		name=SRM,
		description={Einkaufspolitik ist ein Teilgebiet der Unternehmenspolitik, das sich mit der Bestimmung von Zielen des Einkaufs und der Festlegung von Instrumenten zur Zielverwirklichung befasst. Wesentliche Ziele der Einkaufspolitik sind die Sicherung der Versorgung mit dem in quantitativer und qualitativer Hinsicht richtigen Material sowie die Minimierung der damit verbundenen Kosten},
		first={Supplier Relationship Managemen (SRM)},
		long={Supplier Relationship Managemeng}		
}

\newglossaryentry{scm}{
		name=SCM,
		description={Supply Chain Management bezeichnet den Aufbau und die Verwaltung integrierter Logistikketten (Material- und Informationsflüsse) über den gesamten Wertschöpfungsprozess, ausgehend von der Rohstoffgewinnung über die Veredelungsstufen bis hin zum Endverbraucher},
		first={Supply Chain Management (SCM)},
		long={Supply Chain Management}		
}

\newglossaryentry{plm}{
		name=PLM,
		description={Produktlebenszyklus-Management oder kurz PLM ist ein Hebel für eine erfolgreiche Produktentwicklung und ein strategischer Faktor, der im gesamten Unternehmen zum wirtschaftlichen Nutzen beiträgt. Mithilfe von PLM können Unternehmen komplexe, funktionsübergreifende Prozesse steuern und die Arbeit verteilter Teams so koordinieren, dass konsistent und effizient die bestmöglichen Produkte entstehen},
		first={Produkt Lifecycle Management (PLM)},
		long={Produkt Lifecycle Management}		
}

\newglossaryentry{ondemand}{
		name=OnDemand,
		description={On Demand (deutsch "`auf Anforderung"', "`auf Abruf"') ist ein Begriffszusatz für Dienstleistungen, Waren oder Ähnliches, der auf eine zeitnahe Erfüllung von Anforderungen bzw. Nachfragen hinweisen soll. Die On-Demand-Systeme und -Prozesse müssen flexibel angelegt sein, da sie häufig Echtzeitforderungen unterliegen. Zur Erbringung der geplanten Leistung benötigen sie den vollen Zugriff auf die notwendigen Ressourcen. Sie sind daher unter Normalbedingungen leistungsfähiger und höher integriert als Systeme, die ein vergleichbares Endprodukt nicht sofort erbringen\cite{OnDemandDefinition}},
		first={OnDemand},
		long={OnDemand}		
}

\newglossaryentry{onpremise}{
		name=OnPremise,
		description={Als On-Premise wird das traditionelle Modell der Softwarebereitstellung bezeichnet, bei dem ein Unternehmen Softwarelizenzen erwirbt und Anwendungen lokal implementiert und verwaltet. Es handelt sich somit also um eine Vor-Ort-Infrastruktur\cite{OnPremiseDefinition}},
		first={OnPremise},
		long={OnPremise}		
}

\newglossaryentry{byd}{
		name=ByD,
		description={Business By Design},
		first={Business By Design (ByD)},
		long={Business By Design}		
}

\newglossaryentry{saas}{
		name=SaaS,
		description={Software-as-a-Service},
		first={Software-as-a-Service  (SaaS)},
		long={Software-as-a-Service}		
}

\newglossaryentry{sme}{
		name=SME,
		description={Small and medium enterprises},
		first={Small and medium enterprises  (SME)},
		long={Small and medium enterprises}		
}

%glossaries nur Worterklärung
\newglossaryentry{sap}{
name=SAP,
description={Systems Applications Products / Systeme Anwendungen Produkte}
}

\newglossaryentry{ibm}{
name=IBM,
description={International Business Machines Corporation}
}

\newglossaryentry{hana}{
name=HANA,
description={High Performance Analytic Appliance}
}

\newglossaryentry{db}{
name=DB,
description={Datenbank}
}

\newglossaryentry{bzw}{
name=Bzw.,
description={Beziehungsweise}
}

\newglossaryentry{ua}{
name=u.a.,
description={unter anderem}
}

\newglossaryentry{vgl}{
name=Vgl.,
description={Vergleich}
}

\newglossaryentry{zb}{
name=z.B.,
description={zum Beispiel}
}







