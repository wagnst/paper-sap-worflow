%GLOSSAR

%glossaries mit Acronymen
\newglossaryentry{abap}
{
    name={ABAP},
    description={Ist eine Programmiersprache der \gls{sap} AG.},
    first={Advanced Business Application Programming (ABAP)},
    long={Advanced Business Application Programming}
}

\newglossaryentry{ui}
{
    name={UI},
    description={Bezeichnet die Bedienoberfläche eines Computerprogramms.},
    first={User Interface (UI)},
    long={User Interface}
}

\newglossaryentry{nw}
{
    name={NW},
    description={SAP NetWeaver ist ein Produkt der Firma SAP, die NetWeaver als Plattform für Geschäftsanwendungen bezeichnet. Grundlage für alle Anwendungen auf NetWeaver ist der SAP NetWeaver Application Server (siehe \ref{sec:netweaver})},
    first={NetWeaver (NW)},
    long={NetWeaver}
}

\newglossaryentry{erp}{
		name=ERP,
		description={Enterprise Resource Planning (siehe \ref{sec:erp-definition})},
		first={Enterprise Resource Planning (ERP)},
		long={Enterprise Resource Planning}
}

\newglossaryentry{crm}{
		name=CRM,
		description={Customer Relationship Management (siehe \ref{sec:crm-definition})},
		first={Customer Relationship Managment (CRM)},
		long={Customer Relationship Managment}		
}

\newglossaryentry{srm}{
		name=SRM,
		description={Supplier Relationship Management (siehe \ref{sec:srm-definition})},
		long={Supplier Relationship Managemeng}		
}

\newglossaryentry{scm}{
		name=SCM,
		description={Supply Chain Management (siehe \ref{sec:scm-definition})},
		first={Supply Chain Management (SCM)},
		long={Supply Chain Management}		
}

\newglossaryentry{plm}{
		name=PLM,
		description={Product Lifecycle Management (siehe \ref{sec:plm-definition})},
		first={Produkt Lifecycle Management (PLM)},
		long={Produkt Lifecycle Management}		
}

\newglossaryentry{ondemand}{
		name=OnDemand,
		description={On Demand (deutsch "`auf Anforderung"', "`auf Abruf"') ist ein Begriffszusatz für Dienstleistungen, Waren oder Ähnliches, der auf eine zeitnahe Erfüllung von Anforderungen bzw. Nachfragen hinweisen soll. Die On-Demand-Systeme und -Prozesse müssen flexibel angelegt sein, da sie häufig Echtzeitforderungen unterliegen. Zur Erbringung der geplanten Leistung benötigen sie den vollen Zugriff auf die notwendigen Ressourcen. Sie sind daher unter Normalbedingungen leistungsfähiger und höher integriert als Systeme, die ein vergleichbares Endprodukt nicht sofort erbringen (Quelle: \cite{OnDemandDefinition})},
		first={OnDemand},
		long={OnDemand}		
}

\newglossaryentry{onpremise}{
		name=OnPremise,
		description={Als On-Premise wird das traditionelle Modell der Softwarebereitstellung bezeichnet, bei dem ein Unternehmen Softwarelizenzen erwirbt und Anwendungen lokal implementiert und verwaltet. Es handelt sich somit also um eine Vor-Ort-Infrastruktur (Quelle:\cite{OnPremiseDefinition})},
		first={OnPremise},
		long={OnPremise}		
}

\newglossaryentry{byd}{
		name=ByD,
		description={Business By Design (siehe \ref{sec:byd}},
		first={Business By Design (ByD)},
		long={Business By Design}		
}

\newglossaryentry{saas}{
		name=SaaS,
		description={Software-as-a-Service},
		first={Software-as-a-Service (SaaS)},
		long={Software-as-a-Service}		
}

\newglossaryentry{sme}{
		name=SME,
		description={Small and medium enterprises / kleine und mittelständische Unternehmen},
		first={Small and medium enterprises (SME)},
		long={Small and medium enterprises}		
}

%glossaries nur Worterklärung
\newglossaryentry{sap}{
name=SAP,
description={Systems Applications Products / Systeme Anwendungen Produkte}
}

\newglossaryentry{ibm}{
name=IBM,
description={International Business Machines Corporation}
}

\newglossaryentry{hana}{
name=HANA,
description={High Performance Analytic Appliance, Datenbanktechnologie von \gls{sap}}
}

\newglossaryentry{db}{
name=DB,
description={Datenbank}
}

\newglossaryentry{bzw}{
name=Bzw.,
description={Beziehungsweise}
}

\newglossaryentry{ua}{
name=u.a.,
description={unter anderem}
}

\newglossaryentry{vgl}{
name=Vgl.,
description={Vergleich}
}

\newglossaryentry{zb}{
name=z.B.,
description={zum Beispiel}
}







