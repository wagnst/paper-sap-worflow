%%%
%
% haeufig benoetigte Festlegungen
%
%%%

%% Erweiterungspakete werden geladen

% erlaubt direkte Verwendung von Umlauten im Quelltext (latin1)
%\usepackage{umlaut} 
\usepackage[utf8]{inputenc}% http://ctan.org/pkg/inputenc

% länderspezifische Einstellungen
% sorgt u.a. für die korrekte Silbentrennung,
% deutsche Bezeichner ("Tabelle", "Abbildung" etc.)
\usepackage[ngerman]{babel}

% fuer deutsche W"orter in der Literaturliste
\usepackage{bibgerm}  

% Sonderzeichen, z.B. Euro mit \texteuro   
\usepackage{textcomp}         

% erlaubt einfache Auswahl der Art der Nummerierung bei Aufzählungen
\usepackage{enumerate}

% erlaubt das Einbinden von Grafiken
\usepackage{graphicx}

% erlaubt die einfache Aenderung der Seitenraennder
%\usepackage[left=15mm,right=15mm,top=19mm,bottom=19mm]{geometry}

\usepackage[ngerman]{translator}

\usepackage{tabularx}    
                       
\usepackage{booktabs}  

\usepackage{setspace} 

\usepackage{amssymb}

\usepackage{eurosym}

\usepackage{lastpage}

\usepackage{multicol}

\usepackage{pgf-pie,etoolbox}

\usepackage{wrapfig}

\usepackage{float}

\usepackage{listings}

\definecolor{light-gray}{gray}{0.95}
% ------------------------------ Geometrie--------------------------------------
\usepackage{geometry,blindtext}
\geometry{a4paper,left=30mm,right=20mm, top=1cm, bottom=2cm, includeheadfoot}

% ---------------------------- Kopf-Fußzeilen-----------------------------------
\usepackage{fancyhdr}
\pagestyle{fancy}
\fancyhf{}

\fancyhead[L]{\small{\textbf{Proseminar Workflow}}}
\fancyhead[C]{\small{Steffen Wagner, Marco Dörfer, Jonas Dann}}
\fancyhead[R]{\includegraphics[scale=0.1]{grafiken/sap_logo.png}}
\renewcommand{\headrulewidth}{0.5pt} %obere Trennlinie
\fancyfoot[C]{\thepage\ von \pageref{LastPage}} %Seitennummer
\renewcommand{\footrulewidth}{0.4pt} %untere Trennlinie
\def\chapterpagestyle{fancy} %auch bei Seiten mit Chapter Kopfzeile anzeigen


% ----------------------------- Hyperlinks -------------------------------------
\usepackage{breakurl}         % Zeilenumbruch f"ur URLs

% Links zum Anklicken im DVI- und PDF-Dokument
\usepackage{hyperref} 
\hypersetup{colorlinks
  ,linkcolor=blue             % toc, Glossar-Begriffe, Seitenzahlen in Index und Glossar
  ,urlcolor=blue              % URLs, die mit \url{} erzeugt wurden
  ,citecolor=blue             % Literatur-Zitate, die mit \cite erzeugt wurden
  ,filecolor=red              % Verweise auf Dateien, hier nicht verwendet
  ,breaklinks=true            % Zeilenumbruch f"ur Links
  ,linktocpage                % Nur Seitenzahlen sind Links, nicht ganze Zeilen
}
\def\UrlFont{\sffamily} 

% ------------------------------ Glossar ---------------------------------------
\usepackage[
% nonumberlist,               % keine Seitenzahlen anzeigen
acronym,                      % ein Abk"urzungsverzeichnis erstellen
toc,                          % Eintr"age im Inhaltsverzeichnis
section                       % im Inhaltsverzeichnis auf Section-Ebene erscheinen
]{glossaries}                 % definiert den Befehl \printglossary
\renewcommand*{\glspostdescription}{} %Den Punkt am Ende jeder Beschreibung deaktivieren
 
%Ein eigenes Symbolverzeichnis erstellen
\newglossary[slg]{symbolslist}{syi}{syg}{Symbolverzeichnis}

%Glossar-Befehle anschalten
\makeglossaries
 
% ----------------------------- Stichwortverzeichnis ---------------------------
\usepackage{makeidx}          % definiert den Befehl \printindex
\makeindex                    % erzeugt fuenftes.idx f"ur den Index

% ----------------------------- Aussehen einer Seite ---------------------------
%\textheight240mm              % Hoehe des Textes
%\textwidth150mm               % Breite des Textes
%\topmargin-20mm               % oberer Rand
%\oddsidemargin-7mm            % linker Rand bei ungeraden Seitenzahlen
%\evensidemargin-7mm           % linker Rand bei geraden Seitenzahlen
%\pagestyle{plain}             % plain    = Seitenzahlen, aber keine Kopfzeilen
                              % empty    = ohne Seitenzahlen
                              % headings = mit Kopfzeilen
%\parindent0mm                 % kein Einr"ucken am Anfang eines Absatzes


% kein "haengender" Einzug der ersten Zeilen eines Absatzes
\setlength{\parindent}{0cm}

% vertikaler Abstand zwischen Absätzen
% (1ex entspricht der Höhe des Buchstabens x, diese Angabe ist
% relativ zur gewählten Schriftgröße und passt sich somit bei
% einer Änderung entsprechend an)
\setlength{\parskip}{1ex}

